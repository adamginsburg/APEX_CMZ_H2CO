\input{preface}

\begin{document}

\title{APEX CMZ \para 218 GHz Survey: Temperature Map of the CMZ}
\titlerunning{APEX CMZ \formaldehyde}
\authorrunning{Ginsburg et al}
\newcommand{\eso}     {$^{1 }$}
\newcommand{\mpifr}   {$^{2 }$}
\newcommand{\saudi}   {$^{3 }$}
\newcommand{\naoj}    {$^{4 }$}
\newcommand{\pmo}     {$^{5 }$}
\newcommand{\nrao}    {$^{6 }$}
\newcommand{\casa}    {$^{7 }$}
\newcommand{\cfa}     {$^{8 }$}
\newcommand{\chalmers}{$^{9 }$}
\newcommand{\oxford}  {$^{10}$}
\newcommand{\mpa}     {$^{11}$}
\newcommand{\ljmu}    {$^{12}$}
\newcommand{\lmu}     {$^{13}$}

\author{Adam Ginsburg{\eso},
        Christian Henkel{\mpifr,\saudi},
        Yiping Ao{\naoj,\pmo},
        Denise Riquelme{\mpifr},
        Jens Kauffmann{\mpifr},
        Thushara Pillai{\mpifr},
        Elizabeth A.C. Mills{\nrao},
        M. A. Requena-Torres{\mpifr},
        Katharina Immer{\eso},
        Leonardo Testi{\eso},
        Juergen Ott{\nrao},
        John Bally{\casa},
        Cara Battersby{\cfa},
        Jeremy Darling{\casa},
        Susanne Aalto{\chalmers},
        Thomas Stanke{\eso},
        Sarah Kendrew{\oxford},
        J.M. Diederik Kruijssen{\mpa},
        Steven Longmore{\ljmu},
        James Dale{\lmu},
        %Alexandre Faure{}, % invite him first
        Rolf Guesten{\mpifr},
        K.M. Menten{\mpifr}
        }

%\institute{
%      {$^\casa$}{\it{CASA, University of Colorado, 389-UCB, Boulder, CO 80309}}}
%      {$^\eso$}{\it{European Southern Observatory, Karl-Schwarzschild-Strasse 2, D-85748 Garching bei München, Germany}}}
%      {$^\cfa$}{\it{CfA}}}
%      {$^\mpifr$}{\it{Max Planck Institute for Radio Astronomy, auf dem Hugel, Bonn}}}
%      {$^\nrao$}{\it{National Radio Astronomy Observatory, Socorro}}}
%      {$^\oxford$}{\it{Oxford}}}
%      {$^\chalmers$}{\it{Chalmers}}}
%}
\institute{
    {\eso}{\it{European Southern Observatory, Karl-Schwarzschild-Strasse 2, D-85748 Garching bei München, Germany\\
                      \email{Adam.Ginsburg@eso.org}}} \\ 
    {\mpifr}{\it{Max Planck Institute for Radio Astronomy, auf dem Hugel, Bonn}}
    {\saudi}{\it{Astron. Dept., King Abdulaziz University, P.O. Box 80203,
    Jeddah 21589, Saudi Arabia}}\\
    {\naoj}{National Astronomical Observatory of Japan, 2-21-1 Osawa, Mitaka, Tokyo 181-8588, Japan}
    {\pmo}{Purple Mountain Observatory, Chinese Academy of Sciences, Nanjing 210008, China}
    {\nrao}{\it{National Radio Astronomy Observatory, Socorro}}
    {\casa}{\it{CASA, University of Colorado, 389-UCB, Boulder, CO 80309}} \\ 
    {\cfa}{\it{Harvard-Smithsonian Center for Astrophysics, 60 Garden
    Street, Cambridge, MA 02138, USA}} \\ 
    {\chalmers}{\it{Department of Earth and Space Sciences,
                    Chalmers University of Technology}}
    {\oxford}{\it{Department of Astrophysics, The Denys Wilkinson Building, Keble Road, Oxford OX1 3RH}}
    %{\edmonton}{\it{University of Alberta, Department of Physics, 4-181 CCIS, Edmonton AB T6G 2E1 Canada}} \\ 
    %{\yale}{\it{Department of Astronomy, Yale University, P.O. Box 208101, New Haven, CT 06520-8101 USA}} \\ 
    %{\puertorico}{\it{Department of Physical Sciences, University of Puerto Rico, P.O. Box 23323, San Juan, PR 00931}}
    {\mpa}{\it{Max-Planck Institut f\"{u}r Astrophysik, Karl-Schwarzschild-Stra\ss e 1, 85748 Garching, Germany}}
    {\ljmu}{\it{Astrophysics Research Institute, Liverpool John Moores
    University, IC2, Liverpool Science Park, 146 Brownlow Hill, Liverpool L3
    5RF, United Kingdom}}
    {\lmu}{\it{University Observatory Munich, Scheinerstr. 1, D-81679 München, Germany}}
    }


\date{Date: \today ~~ Time: \currenttime}

\abstract
%{The Galactic center is the best region to study star formation under `extreme'
%conditions.  It has a star formation rate below that of normal molecular clouds
%and disk galaxies.}
%{To assess what is suppressing star formation in the Galactic center.}
%{We have mapped the inner 200 pc of the Central Molecular Zone (CMZ) in \para
%in order to measure temperatures in the dense gas.}
%{Gas temperatures in the Galactic Center range from $\sim60$ K to $>100$ K in
%selected regions.  The gas at $R_{gal}>90$ pc is somewhat cooler than that within
%$R_{gal}<90$ pc.}
%{There is a streamer of relatively cool ($T\sim60$ K) gas falling in to the CMZ.
%It overlaps along the line of sight with much of the molecular ring.  Within the
%ring, there is a progression in temperature from Sgr A to Sgr B2 that is
%consistent with proposed orbital models.}
{This is an abstract.}
{To fill it in.}
{Please write your own version of this and send it to Adam.}
{We have filled it in.}
{The abstract is done.}


%\footnote{Compiled on \today\ at \currenttime}

\maketitle

\todo{To-do items are coded in red.}

\section{Introduction}
\todo{The introduction needs to be expanded.}
The central region of our Galaxy is the nearest location in which we can study
star formation in an `extreme' environment.  While there have been great leaps
in our understanding of low-mass star formation from local molecular clouds in
the past decade \citep{Lada2012a,Heiderman2010a,Lada2010a}, there remain many
unanswered questions about how star formation changes as gas becomes denser,
more opaque, and more turbulent, as it most likely was in galaxies at the peak
of cosmic star formation \citep{Kruijssen2013a}.


The central $\sim100$ pc `ring' \citep{Sofue1995a,Molinari2011a} has attracted
a great deal of attention recently, as it has the most gas in the CMZ and a
large, but unrealized, star-forming potential
\citep{Longmore2013a,Longmore2012b,Longmore2012a,Kruijssen2013a,Yusef-Zadeh2009a,Immer2012a}.
The star formation in this region is suppressed relative to expectations based
on nearby star-forming regions and nearby (on a different scale) Galactic disks
\citep{Kennicutt1998a,Kennicutt2012a,Leroy2013a,Heiderman2010a}.

Variations in star formation rate at a fixed gas surface density are observed
in galactic nuclear regions even without AGN.  For example in the nuclei of NGC
2614 and NGC 34, both galaxies with nuclear starburst rings, the star formation
rate is well above the standard star formation laws \citep{Xu2014b}.  However,
in our own Galaxy's circumnuclear ring (not to be confused with the
circumnuclear disk within $\sim1$ pc of Sgr A*), star formation is lower.

Hypotheses for this reduced star formation rate include suppressed
fragmentation by cosmic ray and X-ray heating and a raised star formation
threshold from enhanced turbulence.  \citet{Papadopoulos2010a} and
\citet{Papadopoulos2011a} suggested that high temperatures driven by cosmic
rays may suppress fragmentation.  \citet{Kruijssen2014c} concluded that the
`threshold' for star formation must be higher in the CMZ to explain the
suppressed star formation rate, and therefore turbulence is the essential
mechanism.  

The geometry of CMZ gas is frequently a subject of debate.  While we can in
some cases see absorption signatures indicating the relative line-of-sight
positions of clouds and stars \citep{Longmore2012b,Yusef-Zadeh2012a}, we are
still limited by seeing the CMZ edge-on.  There are known foreground clouds
along the line of sight to the Galactic Center that can be distinguished in
part by their relative line widths, with Galactic Center clouds having typical
linewidths $\sigma_{FWHM}\gtrsim5$ \kms and local clouds $\sigma_{FWHM}
\lesssim 2$ \kms, but this only provides a qualitative distance, distinguishing
nearby ($d\sim1-5$ kpc) from clouds within the central kiloparsec.  Additionally,
there is significant velocity confusion within the Galactic center: many clouds
are seen to overlap in velocity while they are not necessarily at the same distance
\citep{Jones2012a,Oka2012a}

There is a long-standing problem that the observed gas and dust temperatures do
not agree throughout the CMZ
\citep{Guesten1981a,Ao2013a,Ott2014a,Molinari2011a}.  This discrepancy
represents a significant problem for understanding the interplay between gas
temperature and star formation, since the gas temperature drives gas pressure
and the important dynamical effects, while dust temperature is readily
determined on large scales.

These prior works primarily used the popular ammonia (\ammonia) metastable
inversion transition thermometer, which is sensitive to moderate density gas
($n(\hh) \sim A_{ul}/C_{ul} \sim 2\ee{3}$ \percc).  While this thermometer is
generally reliable in cold regions, the population of the higher energy states
- those sensitive to temperatures $T\gtrsim60$ K - may be affected by
`formation heating', a mechanism recently discovered to affect much of the CMZ
gas \citep[][]{Lis2014a,Mills2013a}.  \citet{Ao2013a} used the \para 220 GHz
thermometer \citep{Mangum1993a}, which is sensitive to denser gas, and still
found a significant discrepancy between gas and dust temperature, but they only
covered the inner 30 pc.

We describe new observations using the \para thermometer in Section
\ref{sec:observations}.  In Section \ref{sec:signal}, we describe the analysis
process used to extract signal from the data cubes and derive temperatures.  We
discuss various implications of our data in Section \ref{sec:discussion}.  In
the appendices, we describe further details of the data reduction process,
provide additional tables, and provide the source code for this project.


\section{Observations \& Data Reduction}
\label{sec:observations}

\subsection{Observations}
\todo{Specify rest frequencies.}
We observed the Central Molecular Zone from $-0.4 < \ell < 1.6$ with the SHFI-1
instrument \citep{Vassilev2008a} on the APEX telescope using the XFFTS backend.
The observations were performed in service mode, and were spread out over two
years.  The time was divided into 25 hours in June 2013, 75 hours in April-July
2014, and 50 hours in October 2014.  The time was split between the ESO
(E-093.C-0144A; 50h), MPIfR (M-091.F-0019 and M-093.F-0009; 75h), and OSO
(E-093.C-0144A; 25h) queues.

The 2013 observations were taken in $4\arcmin \times 4\arcmin$ patches, and the
frequency range covered was 217.5-220 GHz and 216-218.5 GHz in the two spectral
windows.  Scans were performed along lines of constant RA and Dec.

The 2014 observing strategy was modified to use larger $8\arcmin \times
8\arcmin$ scans along lines of Galactic latitude and longitude.  The frequency
range was also shifted to cover windows over 217-219.5 and 218.4-220.9 GHz, thus
including the bright \thirteenco and \ceighteeno 2-1 lines.


The raw data were acquired with 32768 spectral channels in each window, yielding
0.1 \kms resolution.  We downsampled the data by a factor of 8 to 0.8 \kms
resolution to make the data more manageable.  We also expect to see no lines
narrower than a few \kms in the CMZ, particularly not with the relatively
shallow observations we have acquired.

Additionally, for the \para data, we incorporated the 41 hours of observations
using the older FFTS backend from \citet{Ao2013a}.  These data covered only 2
GHz of bandwidth, including all three of the \para lines, but they did not
cover the CO lines.

The system temperature ranged from $120 < T_{sys} < 200$ K for the majority of
the observations, with a mean $T_{sys}=165$ K.  A small fraction (10\%) of
observations were in the range $200 < T_{sys} < 300$ K.  There were also a very
small number ($<1\%$) with much higher system temperatures, $300 < T_{sys} <
750$ K, but these are associated with observations of the bright Sgr B2
continuum source.

\subsection{Reduction}
\subsubsection{Calibration}
Calibration was performed at the telescope using the standard APEX calibration
tools
\footnote{See the observing manuals:
\url{www.apex-telescope.org/documents/public/APEX-MPI-MAN-0012.pdf}
\url{www.apex-telescope.org/documents/public/APEX-MPI-MAN-0013.pdf} }.
These yield flux-calibrated spectra at each position with appropriate pointing
information.  Typical flux calibration uncertainties are $\sim10\%$ and
pointing errors $\sigma\lesssim2$\arcsec. % Carlos de Breuck, privat comm

There was a significant calibration error discovered at the APEX telescope during
a large segment of the 2014 observing campaign
(\url{http://www.apex-telescope.org/heterodyne/shfi/calibration/calfactor/}).
The calibration errors were of order $\sim15-25\%$.  They were corrected after
the fact, but the remaining calibration uncertainty
is higher in this data, $\sim15\%$ total rather than the usual $\sim10\%$.

\subsubsection{Flagging bad spectra}
Spectra were removed if they showed excessive noise compared to the
theoretical expectation given the measured system temperature.  As in
\citet{Ao2013a}, the threshold was set to $1.5\times$ the theoretical noise,
i.e. $1.5 \sqrt{2} T_{sys} (\Delta\nu t_{exp})^{0.5}$, where $t_{exp}$ is the
exposure time (integration time) per spectrum in seconds and $\Delta\nu$ is the
channel width in Hz.  This approach resulted in $\sim0.2\%$ of the data being
removed.  

At the position of Sgr B2, the noise was significantly higher due to signal
from the continuum source.  We therefore disabled this flagging in a
$2.5\arcmin$ box around Sgr B2.


% Removing these baselines by
% the usual technique of masking out the emitting spectral region and fitting a
% low-order polynomial to the surroundings is somewhat impractical in the
% Galactic Center, where baseline ripples are on comparable scales to the
% extremely wide spectral lines.

%This method was abandoned: it resulted in negative bowls
% We perform a scan-by-scan baseline subtraction as follows: 
% \begin{enumerate}
%     \item Compute the mean spectrum across the scan
%     \item Identify pixels $>1-\sigma$ above the mean spectrum to be ignored
%         when baseline fitting (this is a simple `line masking' process)
%     \item Fit a line to each spectral channel across the scan
%     \item Smooth the fitted line parameters with a gaussian with with
%         $\sigma=10$ channels, interpolating into the line-masked regions
%     \item Subtract the fitted baselines
% \end{enumerate}
% 
% We therefore adopted a principle component analysis (PCA) baseline removal
% approach in which the data were spectrally smoothed and downsampled by an
% additional factor of 25, for a characteristic frequency scale of 15 MHz.  The
% data had their means subtracted before PCA cleaning so that only time-variable
% baseline variation was removed.  The resulting data were decomposed into
% eigenspectra, of which the 3 most correlated components were then set to zero.
% These components visually match the baseline ripples, and due to the wide
% smoothing (20 \kms resolution), no line information is lost.  The selection of
% 3 components was determined empirically by examining the eigenspectra; a test
% removing the first 10 components showed no significant difference in the
% cleaning but did show signs of extracting large-scale correlated signal.
% 
% \Figure{figures/M-091.F-0019-2013-2013-06-08_PCA_high_diagnostic.png}
% {The first 3 eigenspectra from an observation taken on June 8, 2013.  These
% show the baselines that are removed by PCA cleaning.  They do not include any
% signal from the spectral lines.  \todo{This will be
% replaced with a higher-quality figure}}
% {fig:eigenspectra}{0.5}{0}


\subsubsection{Mapmaking}
The maps were made by computing an output grid in Galactic coordinates with
7.2\arcsec pixels and adding each spectrum to the appropriate pixel\footnote{We
use the term `pixel' to refer to a square data element projected on the sky
with axes in Galactic coordinates.  The term `voxel' is used to indicate a cubic data
element, with two axes in galactic coordinates and a third in frequency or
velocity}.  In order
to avoid empty pixels and maximize the signal-to-noise, the spectra were added
to the grid with a weight set from a gaussian with $FWHM=10\arcsec$, which
effectively smooths the output images from $FWHM\approx28\arcsec$ to
$\approx30\arcsec$.  See \citet{Mangum2007a} for more detail on the on-the-fly
mapping technique used here.  The spectra were averaged with inverse-variance
weighting.


%\todo{To demonstrate the utility of the PCA subtraction approach described in Section
%\ref{sec:baseline}, we show maps of the \formaldehyde \threeohthree line before
%and after PCA subtraction.}

The PPV cubes were created with units of brightness temperature ($T_A^*$).  The
main beam efficiency is $\eta_{mb} = 0.75$ (gain $\sim39$
Jy/K)\footnote{\url{http://www.apex-telescope.org/telescope/efficiency/}}.
These values are noted in the FITS headers of the released data.

The maps achieved a depth of $\sigma=50-80$ mK in 1 \kms channels (Figure
\ref{fig:noisestats}).  The noise is moderately lower (about 15\%) than
expected from the online APEX calculator because our noise measurements are
made in moderately smoothed maps.
%While Figure \ref{fig:noisestats} shows the
%generally good performance of SHFI-1 during our observations, our data are
%affected by significant baseline problems which tend to obscure signal.

%\Figure
%{figures/observing_stats}
%{The $1-\sigma$ noise level in each $8\arcmin\times8\arcmin$ field
%computed by measuring the spectral RMS over a 200 \kms line-free region
%from 218.5 to 218.65 GHz, between the \threetwoone and \threetwotwo lines.
%The curve shows a $t^{-1/2}$ scaling with an arbitrarily set amplitude.  The
%$x$-axis shows the mean pixel weight within each field, which is approximately
%proportional to the exposure time scaled by the system temperature.}
%{fig:noisestats}{0.5}{0}


\subsection{Baselining}
\label{sec:baseline}
The data showed significant baseline structure, leading to large-scale
correlated components in the resulting spectra.  The baselines were removed
after cube generation by first identifying bright regions in the \para
\threeohthree line using the technique described in Section \ref{sec:signal},
masking those out for each other line, then fitting a 7th-order polynomial to
the remaining spectra over the velocity range -150 to 250 \kms and subtracting
it.  Lower order baseline removal left a significant and patterned residual.
After performing this baselining, the signal was re-extracted using the same
method.  A more detailed examination of the baseline removal process is
described in Appendix \ref{sec:baselineappendix}.

\section{Signal Extraction \& Masking}
\label{sec:signal}
We use the method described partially in \citet{Ao2013a} and more thoroughly in
\citet{Dame2011b} to mask the data cubes at locations of significant signal in 
the brightest line. 
A noise map was created by computing the sample standard deviation over a
200 \kms range in which no signal was present.
We use the \para \threeohthree line to create the mask by
the following procedure:

\begin{enumerate}
    \item Smooth the data with a gaussian of width 2 pixels in each direction
        (spatial and spectral)
    \item Identify all pixels with brightness $T_A > 2\sigma$ in the smoothed
        map
    \item Grow the mask from the previous step by 1 pixel in each direction
        (this is known in image processing as binary dilation)
\end{enumerate}

The \para \threeohthree mask was then applied to the \threetwoone and
\threetwotwo cubes created with the same PPV gridding.  There is some overlap
between the \methanol \fourtwotwo line and the \para \threetwotwo line in PPV
space, so we shifted the \para mask to the velocity of the \methanol line in
the \para \threetwotwo cube and excluded all pixels with \para detections.
However, inspection of the \threetwotwo cube revealed that significant chunks
of the \threetwotwo signal were excised by this masking, so all further
analysis excludes the \threetwotwo line.

We found that much of the cube included marginal detections at the original
resolution, FWHM=$30\arcsec$ and $\Delta v=1$\kms.  We therefore created
smoothed cubes, smoothing in the spatial dimensions with a
$\sigma_{FWHM}=33.84\arcsec$ gaussian to achieve a resolution 45\arcsec, and
$\sigma_{v, FWMH} = 3$ \kms in the spectral direction.  The smoothed cubes were
also downsampled by a factor of 2 in the spectral direction.  We then performed
the same signal extraction analysis described above, but with a threshold $T_A
> 3\sigma$.  Integrated intensity maps of both the original and the smoothed
data are shown in Figure \ref{fig:integ303} for the \threeohthree map masked
with the above method in and Figure \ref{fig:integ321} for the \threetwoone map
using the mask created from the \threeohthree data.

\RotFigureTwoAA
{figures/integrated/APEX_H2CO_303_202_masked_moment0}
{figures/integrated/APEX_H2CO_303_202_masked_smooth_moment0}
{Integrated intensity (moment 0) maps of the \para \threeohthree line in the
original (a) and smoothed (b) data.  The cubes were masked prior to integration
using the method described in Section \ref{sec:signal}.}
{fig:integ303}{1}{9.5in}

\RotFigureTwoAA
{figures/integrated/APEX_H2CO_321_220_masked_moment0}
{figures/integrated/APEX_H2CO_321_220_masked_smooth_moment0}
{Integrated intensity (moment 0) maps of the \para \threetwoone line in the
original (a) and smoothed (b) data.  The cubes were masked prior to integration
using the mask generated from the \threeohthree line as described in Section
\ref{sec:signal}.}
{fig:integ321}{1}{9.5in}

\subsection{Ratio Maps}
\label{sec:ratio}
We have created \Rone ratio maps as a first step toward building temperature
maps.  In general, a higher \Rone indicates a higher temperature, and therefore
the ratio maps are a good proxy for \emph{relative} temperature.  We display
four different varieties of ratio map in Figures \ref{fig:ratiomaps} and
\ref{fig:ratiomapssm}.

Figure \ref{fig:ratiomaps} shows maps created by creating ratio cubes, in which
each voxel has the value $R_1=\Rone$, then averaging across velocity with
constant weight for each voxel in Figure \ref{ratiomaps}a and with the weight
set by the \threeohthree brightness value in Fiure \ref{ratiomaps}b.  In both
cases, the mask described in Section \ref{sec:signal} was applied to the cubes
before averaging over velocity.

Figure \ref{fig:ratiomapssm} shows the same process performed on the smoothed
data cubes described in Section \ref{sec:signal}.


%We created two independent ratio maps, \threetwotwo/\threeohthree and
%\threetwoone/\threeohthree, so that the two ratios (which are expected to be
%nearly the same) can be used as a sanity check.  However, in many regions,
%especially those with the greatest signal, the \threetwotwo line is strongly
%affected by contamination from a \methanol line, so that ratio is not
%usable for integrated maps.  The \Rone maps are used for all further analysis.

Ratio maps are subject to dramatic uncertainties in the low signal-to-noise
regime, which is inevitably reached in some parts of the map.  The error
distribution on the ratio becomes non-Gaussian as signal approaches zero,
approaching a pathological Cauchy distribution in which the mean becomes
undefined.
We therefore limited our ratio measurements to regions with significance
$>5\sigma$ in the \threeohthree line and used that line as the denominator
in the ratio.  By using the brighter line as the denominator, we avoid
divide-by-zero numerical errors.


% figure_ratio_maps
\RotFigureTwoAA
%{figures/big_H2CO_322221_to_303202_smooth_bl_integ.pdf}
{figures/big_maps/big_H2CO_321220_to_303202_bl_integ.pdf}
{figures/big_maps/big_H2CO_321220_to_303202_bl_integ_masked_weighted.pdf}
%{figures/big_maps/big_lores_tmap_withcontours.pdf} % tmap_figure
{(a) Map of the ratio of the integrated emission of
\Rone.  The data cubes were masked by
signal-to-noise ratio in the \threeohthree line, with blank (gray) regions
indicating nondetections.  The Sgr B2 peaks exhibit \formaldehyde
absorption and therefore are not reliable.  Higher ratios (red) correspond
to higher temperatures.
\newline
(b) The same as (a), but the average along velocity has been
weighted by the \threeohthree brightness
\newline
}
%(c) The temperature map derived from (a).  The modeling used to derive
%temperature is discussed in Section \ref{sec:linemodeling}.
%{fig:ratiomaps}{1}{7in}
%{fig:ratiomaps}{0.6}{0}
{fig:ratiomaps}{1}{9.5in}

% figure_ratio_maps
\RotFigureTwoAA
%{figures/big_H2CO_322221_to_303202_smooth_bl_integ.pdf}
{figures/big_maps/big_H2CO_321220_to_303202_smooth_bl_integ.pdf}
{figures/big_maps/big_H2CO_321220_to_303202_smooth_bl_integ_masked_weighted.pdf}
%{figures/big_maps/big_lores_smooth_tmap_withcontours.pdf} % tmap_figure
{Same as figure \ref{fig:ratiomaps}, but smoothed with a 34\arcsec gaussian.
(a) Map of the ratio of the integrated emission of
\Rone.  The data cubes were masked by
signal-to-noise ratio in the \threeohthree line, with blank (gray) regions
indicating nondetections.  The Sgr B2 peaks exhibit \formaldehyde
absorption and therefore are not reliable.  Higher ratios (red) correspond
to higher temperatures.
\newline
(b) The same as (a), but the average along velocity has been
weighted by the \threeohthree brightness
\newline
%(c) The temperature map derived from (a).  The modeling used to derive
%temperature is discussed in Section \ref{sec:linemodeling}.
}
%{fig:ratiomaps}{1}{7in}
%{fig:ratiomaps}{0.6}{0}
{fig:ratiomapssm}{1}{9.5in}


\subsection{Line Modeling}
\label{sec:linemodeling}
% TODO: I switched from 15.1 to 17 in column on Jan 14.  This may break all
% of the code when I re-run it because it is changing the underlying prior!
%
We use RADEX \citet{van-Der-Tak2007a} and a related solver (Fujun Du's
\texttt{myRadex}; \url{https://github.com/fjdu/myRadex}) to create a model grid
for the p-\formaldehyde lines over a grid of 20 density ($n=10^{2.5}$-$10^{7}$
\percc), a fixed assumed line gradient of 5 \kms / pc, 20 \formaldehyde column
densities $N(\formaldehyde) = 1\ee{11}-1\ee{15.1}$ \persc, and 50 temperatures
from 10 to 350 K.  The grid was created using the \texttt{pyradex} wrapper of
RADEX\footnote{\url{https://github.com/adamginsburg/pyradex}} and collision rates
retrieved from LAMDA \citep{Wiesenfeld2003a}.  We then
upsampled these grids by spline interpolating between grid points to acquire a
high-resolution ($T\times n \times N = 250\times100\times100$) grid covering
$10<T<350$ K, $10^{2.5} < n < 10^7$ \percc, and $10^{11} < N < 10^{15.1}$
\persc\perkmspc.

We extracted spectra averaged over selected regions and fitted
a 6-parameter model to the full 218-219 GHz spectral range using
\texttt{pyspeckit} \citep{Ginsburg2011c}.  The fitted parameters are the amplitude of the
\formaldehyde \threeohthree line, the centroid velocity, the line width
($\sigma$, not FWHM), the ratio $R_1 = \Rone$, the ratio $R_2 = \Rtwo$, and the
amplitude of the \methanol \fourtwotwo line.  For these spectra, we performed
a different baseline subtraction to that described in Section
\ref{sec:baseline}.  We first fit the described 6-parameter model, then fit a
spline curve to the residual smoothed to a 50-channel (50 \kms) scale.  We then
re-fit the model to the data.  In all but the lowest signal-to-noise cases,
this approach resulted in consistent ratio measurements between the two fits
and with the other methods.  Figure \ref{fig:twospectra_hotcold} shows two example
spectra with significantly different ratios measured and with the spline baseline
fit indicated.

\FigureTwo
{figures/simple/Brick_SW_fit_4_lines_simple_splinebaselined.pdf}
{figures/simple/ds:G0.42+0.03box_fit_4_lines_simple_splinebaselined.pdf}
{
Fits to the \para lines and the \methanol \fourtwotwo line for (a) The Brick
and (b) a box centered on G0.42+0.03.  These spectra have significantly
different ratios and therefore derived temperatures: The Brick has
$T\gtrsim100$ K while G0.42+0.03 has $T\approx40-50$ K.  The bottom black
spectrum is the residual, and the orange wiggly spectrum shows the spline fit
used to remove baseline ripples.}
{fig:twospectra_hotcold}{1}{3in}


For each extracted region, we created $\chi^2$ grids using independent
constraints from the line ratio, the \formaldehyde abundance, the total column
density of \hh, and a lower limit on the \hh volume density.  We use the line
\emph{ratio} rather than line
brightnesses to avoid uncertainties due to the ``filling factor'' of the
emitting gas: even for diffuse clouds, the filling factor of the emitting
regions may be $ff<<1$ if the emission is isolated to compact shocked regions,
as expected if highly supersonic $\mathcal{M}>10$ turbulence is energetically
dominant in the clouds.  We use an abundance $N(\para)/N(\hh) = X(\para) =
10^{9.08\pm1} = 1.2\ee{9} \times/\div 10$, allowing for dramatic uncertainty in
the \formaldehyde abundance
\citep{Ginsburg2013a,Carey1998a,Wootten1978a,Mundy1987a}.  To constrain the
total column density, we use the Herschel dust maps to derive an \hh column
density, which has a nominal $\sim2-3\times$ uncertainty.  We treat the error
as $10\times$ to account for the abundance uncertainty.  This large assumed
error also accounts for factor $\sim2-3$ uncertainty due to line-of-sight
confusion, since the dust column density cannot be associated with any specific
velocity component.  To constrain the dust density, we assume the selected area
has a mass given by the Herschel dust-derived mass and a spherical volume,
which sets a conservative (but nonetheless highly constraining) lower limit on
the volume density.

These constraints are shown projected onto the three planes of our fitted
parameter space in the multi-paneled Figures
\ref{fig:coltemconstraints}-\ref{fig:parsonbrightness}.  The fitted parameters
are displayed in Figure \ref{fig:parsonbrightness}.

While $R_2$ can, in some cases, provide significant constraints on various
parameters, we do not use it because of the ambiguity imposed by the overlap
with the \methanol \fourtwotwo line.

% TODO: bootstrap-ish analysis?  For a given *known* set of physical parameters,
% what will we derive?
%We note that $\chi^2$ minimization, while commonly used for parameter space
%fitting with LVG grid \citep[e.g.][]{Zeiger2010a,Ginsburg2011a,Ao2013a}, is not
%well justified.  The $\Delta \chi^2$ values probably do not correspond to their


% individual_spectra
\Figure{figures/simple/c:G0.38+0.04_fit_4_lines_simple.pdf}
{Fitted spectrum of ``cloud c".  The fitted parameters and their corresponding
errors are shown in the legend.  The parameters are the amplitude of the
\formaldehyde \threeohthree line, the centroid velocity, the line width
($\sigma$, not FWHM), the ratio $R_1 = \Rone$,
the ratio $R_2 = \Rtwo$, and the amplitude of
the \methanol \fourtwotwo line.  }
{fig:cloudcspec}{0.5}{0}

% param_plots
\Figure{figures/param_fits/c:G0.38+0.04_col_tem_0_parameter_constraints.pdf}
{The parameter constraints for ``cloud c'' (Figure \ref{fig:cloudcspec})
projected (marginalized) onto the temperature/column density plane.\newline
(top left) Constraints imposed by the measured ratio \Rone are shown in filled contours,
with blue corresponding to $\Delta\chi^2 < 1$, cyan $1 < \Delta\chi^2 < 2$,
yellow $2 < \Delta\chi^2 < 3$, and red $3 < \Delta\chi^2 < 4$.  The colored line
contours show the joint constraints imposed by including restrictions on the
total column density, volume density,  filling factor, and abundance, following
the same color
scheme.  The solid grayscale contours show the constraints imposed by
$R_2=\Rtwo$, from black ($\Delta\chi^2 < 1$) to white.  In this case, $R_2$
is weakly inconsistent with the joint constraints from the other parameters.
\newline
(top right) The same colorscheme as before, showing the constraints imposed by
assuming the abundance of \para relative to \hh is as labeled.  
The abundance does not constrain these parameters.
\newline
(bottom right) The same colorscheme as before, showing the constraints imposed
by using the measured mean volume density as a lower limit.
\newline
(bottom left) The same colorscheme as before, showing the constraints imposed
by the measured total column density of \hh, converted to a constraint on the
\para column by assuming a velocity gradient (5 \kms \perpc) and abundance as
shown in the top right panel.  The uncertainty is dominated by the abundance
uncertainty.
}
{fig:coltemconstraints}{0.5}{0}

\Figure{figures/param_fits/c:G0.38+0.04_dens_col_0_parameter_constraints.pdf}
{The constraints in density-column parameter space.
See Figure \ref{fig:coltemconstraints} for details.
}
{fig:denscolconstraints}{0.5}{0}

\Figure{figures/param_fits/c:G0.38+0.04_dens_tem_0_parameter_constraints.pdf}
{The constraints in density-temperature parameter space.
See Figure \ref{fig:coltemconstraints} for details.}
{fig:denstemconstraints}{0.5}{0}

\Figure{figures/param_fits/c:G0.38+0.04_0_h2coratio_minaxis.pdf}
{The line brightness of \para \threeohthree (top row) and \para \threetwoone
(bottom row) in the three different projections of parameter space.  The
grayscale images correspond to a slice through the parameter spaces at the
location of the best-fit parameter.  The colored contours show the allowed
marginalized regions in each parameter space as described in the
Figure \ref{fig:coltemconstraints} caption.}
{fig:parsonbrightness}{0.5}{0}

\section{Region Extraction}
\label{sec:region}
In order to extract higher signal-to-noise measurements on selected regions, we
broke the data down into subsets using both a by-eye region selection drawing
regions using ds9 and a more systematic approach using a dendrogram-based clump
finding algorithm \citep[][\url{http://dendrograms.org/}]{Rosolowsky2008c}.

% dendro_temperature.py
% TODO: decide whether we used the full resolution or the smoothed version.
% Presently (Jan 5), we are using the smoothed version
We ran the dendrogram/clumpfind extraction analysis on the smoothed
\threeohthree cube data.  We enforced thresholds of a minimum number of pixels
$N_{min}=100$, a peak threshold of $I = 3\sigma$, and a splitting threshold
$\Delta = 2\sigma$.  The exact values of these parameters is not particularly
important, as we are interested in general trends with size-scale and
galactocentric distance, but we caution against overinterpretation of the
resulting catalog as the number of sources and their size and distribution can
change dramatically with small changes in the selected parameters.  The number
and location of pixels included in the catalog is, however, relatively robust
against small parameter changes.

For each extracted blob, we measured the corresponding integrated and peak
emission in the \threetwoone and \thirteenco cubes.  We also extracted the mean
dust temperature and column density from SED fits to Herschel HiGal 170-500\um
maps.  The SED fits were performed on background-subtracted data using an
approach originally described in \citet{Battersby2011a}, which includes details
of the fitting process and specification of assumed physical parameters.  We
also extracted dust column density and temperature using a more naive
pixel-by-pixel approach with no background subtraction
(\url{http://hi-gal-sed-fitter.readthedocs.org/en/latest/}) and found the two
to be consistent.

The by-eye selected regions were extracted less systematically, instead
focusing on peaks in the \threeohthree or \threetwoone emission.  We also
included a series of large square $8\arcmin \times 8\arcmin$ `apertures'
corresponding to each observed field.  For each region extracted in this
fashion, we fit a 1-3 component model as described in Section
\ref{sec:linemodeling}.  An example spectrum extracted from a small region
around ``cloud c'' is shown in Figure \ref{fig:cloudcspec}.


\subsection{Blob modeling}
\label{sec:dendromod}
Similar to the line modeling approach, we derived temperatures for each
dendrogram catalog blob using the measured line ratio, dust column density, 
and a dust-derived lower limit on the volume density as constraints.
The $\chi^2$ grid approach described in Section \ref{sec:linemodeling} was used
to extract the temperature and associated error.

We plot the derived temperature for all clumps against the measured line ratio
in Figure \ref{fig:ratiovstem}.  The figure illustrates the effects of assuming
higher or lower density on the temperature. The lower density models are
excluded by the observed volume density lower limit from the dust.

% dendrotem_plots
\Figure{figures/dendrotem/ratio_vs_temperature_modeloverlay.pdf}
{The derived temperature vs the measured ratio \Rone for the dendrogram clumps.
The big symbols represent leaves (compact clumps) and the small symbols
represent larger parent structures.
The temperature includes constraints from the assumed fixed \formaldehyde
abundance and the 
varying column density.  The points are color coded by signal-to-noise in the
ratio $R_1$, with black $S/N < 5$, blue $5 < S/N < 25$, green $25 < S/N < 50$,
and red $S/N > 50$.  The black lines show the modeled temperature as a function of
$R_1$ for three different assumed densities with a constant assumed abundance.
}
{fig:ratiovstem}{0.6}{0}

\subsection{Temperature Maps}
\label{sec:formaldehydetemmap}
Using the relation described in Section \ref{sec:dendromod}, we
converted the ratio map shown in Figure \ref{fig:ratiomaps} to a temperature
map in Figure \ref{fig:temmap}a.  Because this map includes the full line-of-sight
integrated emission, there are many regions where multiple independent components
are being mixed.

%Figure \ref{fig:temmap}b shows a temperature map created by measuring the
%temperature in each clump from the dendrogram clump extraction based on the
%smoothed data cube and averaging over the velocity axis.  Because there are
%hierarchically nested clumps in the dendrogram-based clumpfind, and because
%there are genuinely independent clumps along common lines of sight, there is
%still overlap and confusion.  The nested clumps in the dendrogram can provide
%more than one temperature measurement per voxel; we therefore use the most
%compact clump's temperature measurement for each voxel.

%The two temperature maps broadly agree, but the dendrogram map is smoother and
%aesthetically superior.  The noisy edges of regions in Figure \ref{fig:temmap}a
%are unlikely to be real temperature features, so the dendrogram map is
%preferred.

These temperature maps do not have uniform reliability.  There
is some spatial variation in the noise, but the variation in the signal is much
more important.  In regions where the peak signal is low, the reliability of
these maps is substantially reduced.  Figure \ref{fig:peaksn} shows the peak
signal-to-noise in both the unsmoothed and the smoothed maps to indicate where
the temperature measurements can be relied upon.  We have also applied a mask
to both panels of Figure \ref{fig:temmap} to make the low-reliability regions
blend into the background by adding a foreground gray layer with opacity set by
the inverse S/N from Figure \ref{fig:peaksn}b.

The effective range of detected temperatures is $\approx40-150$ K. While lower
and higher temperatures appear in the map, those measurements should not be
trusted.  Below $T\lesssim40$ K, all temperature measurements are effectively
upper limits, generally corresponding to $T\lesssim40$ K as shown.  This limit
is due to a combination of sensitivity and excitation: below 40 K, the \para
lines are observed to be faint, and therefore the \threetwoone line is not
detected.  If the high column density gas in, e.g., G0.253+0.016 had a lower
gas temperature, our data would be sensitive enough to measure temperature down
to $T\sim20$ K.  The higher temperatures become lower limits above $\sim150$ K
because the \para line ratio is intrinsically insensitive to higher
temperatures as can be seen in Figure \ref{fig:parsonbrightness}.


% tmap_figure
%\RotFigureTwoAA
\RotFigureThreeAA
%{figures/big_maps/big_lores_smooth_tmap_withcontours.pdf}
{figures/big_maps/big_lores_smoothdens1e4_weighted_masked_tmap_withcontours.pdf}
{figures/big_maps/big_lores_smoothdens3e4_weighted_masked_tmap_withcontours.pdf}
{figures/big_maps/big_lores_smoothdens1e5_weighted_masked_tmap_withcontours.pdf}
{Temperature maps using the smoothed data assuming fixed $X_{\para}=1.2\ee{9}$
and $n(\hh)$.  The maps are masked by signal-to-noise as in Figure
\ref{fig:ratiomaps}.
(a) $n(\hh) = 10^4$ \percc
(b) $n(\hh) = 10^{4.5}$ \percc
(c) $n(\hh) = 10^5$ \percc
}
{fig:temmap}{1}{9.5in}


% % tmap_figure
% %\RotFigureTwoAA
% \RotFigureThreeAA
% %{figures/big_maps/big_lores_smooth_tmap_withcontours.pdf}
% {figures/big_maps/big_RatioCube_DendrogramObjects_smooth_Piecewise_mean.pdf}
% {figures/big_maps/big_RatioCube_DendrogramObjects_smooth_Piecewise_weightedmean.pdf}
% {figures/big_maps/big_lores_smoothdendro_tmap_withcontours.pdf}
% {(top) The mean ratio along each line of sight through the dendrogram-extracted
% cube
% \newline
% (middle) The same as the top figure, but weighted by \threeohthree brightness
% \newline
% (bottom) 
% %A temperature map derived from the \Rone ratio map shown
% %in Figure \ref{fig:ratiomaps}.
% A temperature map created by averaging the dendrogram-derived clump temperature
% along each line-of-sight.
% Regions of lower signal-to-noise, and therefore less
% reliable temperature, are grayed out with a filter that gets more opaque
% toward lower signal-to-noise.  Temperatures above $\gtrsim150$ K should be
% treated as lower limits in the 100-150 K range.
% The thin contours are from the Herschel HiGal dust SED fit at a level $N(\hh)=5\ee{22}$
% and are included to provide a visual reference for comparison between the temperature
% maps.
% %The contours correspond to the integrated \threeohthree
% %line intensity at levels [4,7,11,20,38] K \kms.
% }
% {fig:dendrotemmap}{1}{9.5in}

% tmap_figure
\RotFigureTwoAA
{figures/big_maps/big_lores_peaksn}
{figures/big_maps/big_lores_smooth_peaksn}
{Map of the peak signal-to-noise in the \para \threeohthree line with no
smoothing (top) and with 2-pixel (14.4\arcsec) smoothing (bottom).  
These maps give an indication of the reliability of the temperatures extracted
in Figure \ref{fig:temmap}.  The colorbars are intentionally saturated at a
S/N$>10$ since above this threshold, the temperatures are reliable as long as
they are in the $T<150$ K regime.
}
{fig:peaksn}{1}{9.5in}



% tmap_pv
\RotFigureTwoAA
{figures/big_maps/pv_tmap_weight_masked.pdf}
{figures/big_maps/pv_peaksn.pdf}
{(a) A position-velocity diagram made by averaging the ratio cube along
Galactic latitude and using the $n=10^4$ \percc curve from Figure
\ref{fig:ratiovstem}.  (b) The peak signal-to-noise along Galactic latitude,
which is used as a mask in (a).
\todo{The excessive whitespace in panel (a) is a bug that has not yet been
resolved but it is known.}
}
{fig:pvtem}
{1}{9.5in}

\RotFigureTwoAA
{figures/big_maps/pv_tmap_smooth_weight_masked.pdf}
{figures/big_maps/pv_peaksn_smooth.pdf}
{Smoothed version of Figure \ref{fig:pvtem}.
\todo{The excessive whitespace in panel (a) is a bug that has not yet been
resolved but it is known.}
}
{fig:pvtemsm}
{1}{9.5in}

% t_vs_r
%\Figure
%{figures/temvslon/temperature_vs_longitude_bmean_dendsm}
%{A plot of the latitude-averaged temperature in each pixel as a function of
%longitude.  This plot shows the temperature distribution across the CMZ and is
%meant to better-quantify Figure \ref{fig:temmap}.  The pixels are colored by
%velocity in 20 \kms velocity bins.}
%{fig:tempoints}{0.5}{0}

\section{Discussion}
\label{sec:discussion}
\subsection{Parameter Comparison}
In a turbulent medium, the 1D line width of the gas should be tightly
correlated with the 3D Mach number \citep[e.g.][]{Federrath2011a}.  If
turbulence is the dominant gas heating mechanism, there should also be a
correlation between the Mach number of the turbulence and the gas temperature.
However, Figure \ref{fig:temvsfwhm_regions} shows no such correlation.  For
dendrogram-extracted regions, which consist of multiple nested clumps, there
are trends between the linewidth and temperature, but not with the naively
expected sign.  These trends are likely due to variation in the `clump' size.

\FigureTwo
{figures/chi2_temperature_vs_linewidth_fieldsandsources}
{figures/dendrotem/temperature_vs_rmsvelocity_smooth} % dendrotem_plots
{(a) 
The fitted temperature as described in Section \ref{sec:linemodeling} plotted
against the fitted line width for the by-eye selected regions.  The blue
symbols are compact `clump' sources and the red symbols are large-area
($8\arcmin$) square regions.  No trend between the line width and temperature
is obvious.  The very broad line regions, FWHM $\gtrsim$ 40 \kms, are affected
by significant baseline issues, and it is possible that the temperature has been
underestimated as a result.
(b) The same as (a) but for the dendrogram-extracted sources.  The arc of points
from $\sigma_v=12$ to $8$ \kms and $T=80$ to $140$ K is from the hierarchy of
Sgr B2 clumps and thus represents a variation in temperature with size scale.
Similarly, the other high-linewidth points represent large interconnected
structures.  In the range $1 < \sigma_v < 5$ \kms, which is the only region
including low temperatures, no trend is observed.  The points are color coded
by signal-to-noise in the ratio $R_1$, with black $S/N < 5$, blue $5 < S/N <
25$, green $25 < S/N < 50$, and red $S/N > 50$.  }
{fig:temvsfwhm_regions}
{1}{3.5in}
%{0.5}{0}



\subsection{Variation with Spatial Scale}
For any given cloud, the ratio \Rone varies with the spatial averaging scale.
The \para \threeohthree line is often stronger relative to \threetwoone on
larger scales, while the linewidth is always larger on larger scales.  We have
examined this effect by selecting a few local \para \threeohthree peaks and
measuring the ratio \Rone as a function of aperture size.  For each spectrum,
we have fitted the 6-parameter model described in Section
\ref{sec:linemodeling} to derive the ratio.

Different variations with scale are observed.  For example, around `The Brick',
the ratio decreases from $\sim0.55$ at the position $\ell=0.237$, $b=0.008$
within a single spectrum to $\sim0.33$ when averaged over a radius
$\sim70\arcsec$ (Figure \ref{fig:brickradial}a).  By contrast, Figure
\ref{fig:brickradial}b shows a highly turbulent example where the temperature
is lowest at the center and higher on larger scales.  These observations imply
that two different heating mechanisms, one internally driven (The Brick) and
one external (G1.23-0.08), are active within the CMZ.

% analysis/multiscale_fit
\FigureTwo
{figures/brick_examples/brick_sw_specfit_ratio_vs_scale.pdf}
{figures/g1.2/g1.2_coolspot_ratio_vs_scale.pdf}
{(a) Radial plots centered on the southwest portion of `The Brick'.
The top panel shows the ratio \Rone as a function of aperture size, starting
from a single pixel.
The derived temperature using Figure \ref{fig:ratiovstem} is shown on the right
axis; temperatures above $\gtrsim300$ K are extrapolations from the model and
should be regarded highly skeptically, and temperatures $\gtrsim150$ K should
be regarded as lower limits.
The bottom panel shows the line width $\sigma = FWHM/2.35$
as a function of aperture size.  
(b) The same as (a) but for a region centered on G1.23-0.08, a local
peak in the \para \threeohthree emission.  In this case,
the ratio \Rone is lower at the center and increases toward larger radii.
}
{fig:brickradial}
{1}{3.5in}
%{0.6}{0}



\subsection{Disagreement between gas and dust temperatures}
Similar to the results of \citet{Ao2013a} for the inner $\sim30$ pc, we find
that the \para and dust temperatures do not agree anywhere within the inner
$\sim200$ pc.  Figure \ref{fig:temvstem_regions} shows the fitted \para
temperature vs the fitted dust temperature from HiGal.  The majority of the
\formaldehyde data points are well above the $T_{dust}=T_{gas}$ line
independent of the method used to extract the temperature.  The small handful
of data points near the $T_{dust}=T_{gas}$ line are the low signal-to-noise
points with untrustworthy fits.

\FigureTwo
{figures/chi2_temperature_vs_higaltemperature_fieldsandsources.pdf} % parameter_comparisons
{figures/dendrotem/temperature_vs_dusttem_smooth.pdf} % dendrotem_plots
{(a) The fitted temperature as described in Section \ref{sec:linemodeling} plotted
against the HiGal fitted dust temperature.  As in Figure \ref{fig:temvsfwhm_regions},
the blue symbols are compact `clump' sources and the red symbols are large-area
square regions.  The black dashed line shows the $T_g = T_d$ relation.  Nearly
all of the data points fall above this relation, and very few are consistent
with it at the 3$\sigma$ level.
(b) The same as (a) but for the dendrogram-extracted sources.
The points are color coded by signal-to-noise in the
ratio $R_1$, with black $S/N < 5$, blue $5 < S/N < 25$, green $25 < S/N < 50$,
and red $S/N > 50$.  }
{fig:temvstem_regions}
{1}{3.5in}
%{0.5}{0}

\subsection{Regional variations in gas temperature}
There are significant regional variations in the gas temperature.  The most
obvious is in and around the Sgr B2 complex, where very high gas temperatures
are measured.  High gas temperatures, $T>50$ K, were measured in a ridge to the
Galactic northeast of Sgr B2 N using \methylcyanide previously \citep[][Figure
4b]{de-Vicente1997a}, which we now confirm.  \citet{Ott2014a} also showed high
\ammonia temperatures in this hot ridge.

The high temperatures in the Sgr B2 region, $T>80$ K, extend to a radius of
$\sim10$ pc (5\arcmin), and nearly 30 pc (12.5\arcmin) in the
northwest/southeast directions.  A high temperature `ridge' follows along the
positive-latitude component of the \citet{Molinari2011a} `ring', tracing
through The Brick into the Sgr A complex.

By contrast, lower temperatures, $T\sim60$ K, are observed in the
negative-latitude component of the `ring'.  At positive longitudes,
$\ell\gtrsim0.9$, the gas is uniformly $T\sim60$ K.  Similarly, most of the gas
in the Sgr C complex is around 60 K.

Figure \ref{fig:kdlorbit} shows the orbital fit from \citet{Kruijssen2014d} and
a position-velocity slice through the fitted temperature cube following that
orbit.  The fit is imperfect, especially along the green segment, but still very useful
for tracking the properties of clouds in common environments.
Intriguingly, the `back-side' clouds (green segment), are cooler than
many of those on the front side, such as The Brick (red segment).  Nontheless,
many of the clouds besides The Brick within the molecular `ring' are among the
coldest detected in our survey.

Perhaps the most interesting feature of the orbital extraction is the qualitative
difference between the `big clouds', Sgr B2, Sgr A, and The Brick, and the rest
of the gas.  At least in terms of gas temperature, The Brick is in the same class
as the massive star-forming Sgr B2 complex and the clouds closest to the central
black hole.

There is some evolution of temperature over time evident in Figure
\ref{fig:kdltemvstime}, which shows the temperature at each voxel within 15
\kms and 6 pc of the \citet{Kruijssen2014d} orbit extracted from the
dendrogram-based temperature cube.  The linear increase in temperature seems to
support the \citet{Kruijssen2014d} hypothesis that the `ring' represents a
single gas stream on its first passage by Sgr A*, though The Brick stands out
as a significant outlier in this data.  The Brick has had the most recent
pericenter passage, so it is possible that it is presently overheated from
additional turbulence driven by this most recent passage.  In that case, the
overall temperature rise observed from Sgr A to Sgr B2 cannot be caused by
turbulence decay alone, since The Brick should follow the general trend.
Perhaps, instead, this temperature increase is driven by increased star
formation and internal heating along the dust ridge.

\RotFigureTwoAA
{figures/orbits/KDL2014_orbit_on_H2CO_TemperatureFromRatio_weighted.pdf}
{figures/orbits/KDL2014_orbitpath_on_H2CO_TemperatureFromRatio_weighted.pdf}
{(a) A position-velocity slice through the dendrogram clumpfind generated
temperature cube along the \citet{Kruijssen2014d} orbit.  White regions are
outside the mapped area, while grey regions have no gas detected or inadequate
signal to measure a temperature.
(b) The path used to create the slice shown in (a).  Each element is a small
rectangle over which the temperature has been averaged.  The green rectangle at
$\ell=359.3$ indicates the starting point (offset$=0$) of the position-velocity
slice.  The orbit is outside of our observed field to the west of Sgr C.  
The X's mark The Brick (purple), cloud d (blue), and cloud e (red).
}
{fig:kdlorbit}{1}{8.5in}

\Figure
{figures/orbits/dendro_smooth_temperature_vs_time_firstMyr.pdf}
{The observed temperature vs time since the most recent pericenter passage of
The Brick on the \citet{Kruijssen2014d} orbit.  There is a hint of 
a linear increase in temperature with time, from an initial temperature
$T\sim40-60$ K to $T\sim120$ K over a period $\tau\sim0.5-1$ Myr.}
{fig:kdltemvstime}{0.4}{0}

\subsection{Comparison to \citet{Ott2014a} \ammonia temperature map}
\label{sec:ammoniacompare}
\citet{Ott2014a} measured the \ammonia 1-1 and 2-2 lines over a subset of the
region we mapped, from $-0.2 < \ell < 0.8$.  These two lines are a frequently
used thermometer sensitive to gas temperatures $5 \lesssim T \lesssim 40$ K,
with weak sensitivity up to $\sim80$ K \citep[][Figure 1]{Mangum2013a}.  The
\ammonia lines also have lower critical densities by about 2 orders of
magnitude.  The temperatures
derived from this line ratio are necessarily lower than those we have measured.
Nonetheless, we compare our temperature map to that of \citet{Ott2014a} to look
for common trends and significant differences.

Figure \ref{fig:nh3tmap} shows the \citet{Ott2014a} temperature map in a
similar color scheme to that we have used in Figure \ref{fig:temmap}, but with
different limits.  There are some trends that are common between the two maps:
the Sgr A complex and Sgr B2 complex are the warmest regions, the `diffuse' gas
surrounding these is somewhat cooler, and the positive latitude component of
the \citet{Molinari2011a} `ring' is warmer than the negative latitude
component.

The most significant difference occurs in the southwest portion of the Sgr B2
cloud.  In this region, the \ammonia temperature transitions from the high
($T>80K$) temperatures of the Sgr B2 N hot ridge into a cooler background
$T\sim30-40$ K.  By contrast, the \formaldehyde temperatures stay high,
indicating gas temperatures $T>150$ K extending $\sim2.5\arcmin$ (6 pc) to the
southwest of Sgr B2 S.  This difference probably indicates that there is a
massive, cool, diffuse component being detected in \ammonia, while
\formaldehyde is detecting a hotter, denser component.

\RotFigureTwoAA
{figures/big_maps/ott2014_nh3_tmap_15to150_withcontours.pdf}
{figures/big_maps/lores_smoothdens1e4_weighted_masked_tmap_withcontours.pdf}
{(a) A temperature map derived from the \ammonia 1-1/2-2 line ratio from
\citet{Ott2014a}.  The \ammonia thermometer has a more limited temperature
range when using only these two lines, so the map is cut off at $T=80$ K.
%This map can be compared to Figure \ref{fig:temmap}.
(b) The \para-derived temperature map over the same region from Figure
\ref{fig:temmap}a
}
{fig:nh3tmap}
{1}{9.5in}
%{0.5}{0}

\subsection{Comparison to other measurements with \para}
\label{sec:h2cocompare}
The largest survey of \para temperatures prior to this work was
by \citet{Ao2013a}, who observed the inner 30 pc.  They performed a similar
analyis to ours but over a limited area.  They reported temperatures
systematically lower by 10-20 K (but still $T\gtrsim50$ K).  This difference is
due to a different set of collision rates adopted (\citet{Wiesenfeld2013a}
rather than \citet{Green1991a}) and different priors on the fitted parameter
space.  In particular, we left density as a free parameter and instead broke
degeneracies with a conservative but nonetheless useful constraint on the
column density and abundance.

The \citet{Ao2013a} survey found uniformly high temperatures within the bright
Sgr A cloud complex.  Our survey finds a greater variety in gas temperatures,
with significant contrast between the back-side cool clouds and the warmer
foreground infrared dark clouds.

There have also been observations of \para temperatures with interferometers.
\citet{Johnston2014a} noted a very high temperature peak, $T>320$ K, using these
lines of \para toward The Brick, G0.253+0.015.  We measure similar line
ratios toward The Brick, but on larger scales.  The interpretation of these
line ratios is subject to some question, though.  Above 300K, the collision
rates provided by LAMDA \citep{Green1991a,Schoier2005a,Wiesenfeld2013a} must be
extrapolated.  Additionally, some of the observed ratios in our cannot be
reproduced by high temperatures with the current suite of collision rates, but
instead require that the lines be optically thick and thermalized and therefore
at very high densities.   Given the low brightness temperatures observed, if
the emission is optically thick, it must be coming from an extremely small
area, which is inconsistent with the observed large extent of the emission
unless it is also extremely patchy.  That remains a possibility if we are
observing massive, dense clouds with relatively uniform turbulent properties,
though it is unlikely - we discuss this further in Section
\ref{sec:thickorwarm}.  We conclude, therefore, that a lower limit on the
gas temperature in The Brick, $T>120$ K, is consistent with our observations.


\subsection{Discussion of Heating Mechanisms}
\citet{Ao2013a} examined four heating mechanisms in the inner $\sim 40$ pc,
concluding that the only viable heating mechanisms capable of explaining the
high observed temperatures in the molecular gas are cosmic ray and turbulent
heating.  

The cosmic ray heating rate required by the \citet{Ao2013a} analysis is high,
$\zeta\sim1-2\ee{-14}$ \pers, but plausible.  The turbulence in the CMZ is also
enough to explain the observed gas temperatures. 
The analysis performed in \citet{Ao2013a} determined the gas temperature
assuming that the observed temperature corresponds to an equilibrium between
heating by a single mechanism for all of the gas mass.  For cosmic ray-driven
heating, this is a reasonable approximation: the CR-derived gas temperature
has a weak density dependence ($T_{kin}\sim n^{1/6}$) and the energy deposited
by cosmic rays should be relatively uniform per molecule.  Turbulent heating,
on the other hand, is nonuniform \citep{Pan2009a}.

Non-uniform heating of the molecular gas could explain the temperature
structure observed in the CMZ molecular clouds.  If supersonic shocks within
the gas are responsible for heating the gas above a mean background temperature
$T_{gas} \sim T_{dust}$, only about 10\% of the gas needs to be heated to
$T\sim50-100$ K to generate the line brightness observed \emph{if} the hottest
gas is also the densest.  However, hot and dense gas cools more efficiently, so
sustaining a population of shock-heated gas may be infeasible.  Additionally,
the log-poisson distribution followed by a multiplicative cascade has a substantial
low-temperature tail \citep{Pan2009a}, which implies that the mean temperature
must be near the observed temperature.  \todo{It would be nice to have some help
from a theorist to assess whether non-uniform heating could really drive the observed
temperatures.}

%Gas temperatures higher than dust temperatures occur in both high and low density
%gas \todo{need to have somes sort of density tracer - probably just dust + spherical}


\subsection{Radial Variations}
There are many reasons to expect that gas temperature will increase as gas
falls toward the Galactic center.  As it falls further in on an elliptical but
unclosed orbit, the gas is subject to much greater shear near pericenter,
driving stronger turbulence.  The stellar density increases, implying a
stronger radiation field.  There may be a
greater density of young stars, which in turn would imply more supernovae and a
greater cosmic ray density.  The density of X-ray emitting sources (e.g.,
stellar remnants) also increases, providing yet another source of heating.
Finally, the central black hole, while inactive now, may have gone through
previous bursts of activity, driving brief but powerful waves of X-ray heating.
Independent of the mechanism, though, there is reason to expect some radial
variation in temperature.

Figure \ref{fig:temvsrad} shows the latitude-averaged temperature as a
function of projected radius.  While there is no clear radial trend, there are
important features: the majority of the gas at $R>0.9\arcdeg$ (130 pc) is below
$T\lesssim 60$ K, while within $R<0.9\arcdeg$ the majority of the gas is
between 70 and 120 K (with many lower limits in Sgr B2).  There are hints that
the cooler gas that is observed in this range may actually be at larger
galactocentric radii and simply be projected onto the $R=0$ region.

\Figure
{figures/temvslon/temperature_vs_radius_bmean_dendsm}
{A plot of the latitude-averaged temperature in each pixel as a function of
projected radius. As in Figure \ref{fig:tempoints}, the pixels are colored by
velocity in 20 \kms velocity bins.}
{fig:temvsrad}{0.5}{0}


\subsection{A stream of fresh gas falling into the CMZ?}
\label{sec:coolstream}
The gas streamer extending from $\ell=0.2$ to $\ell=1.3$ (Figure
\ref{fig:coolstream}) may represent a stream of infalling gas.  The lack of
IRDC features corresponding to this stream within $R<R_{SgrB2}$ restricts it to
be on the back side of the CMZ, while the disagreement with the
\citet{Kruijssen2014d} orbit suggests that it
is not following the same path as the Brick, the lettered clouds, and Sgr B2.
The large observed linewidths imply that the stream is genuinely associated
with the CMZ, though its distance from Sgr A may be much larger than its
projected distance.  The path we have traced through Figure \ref{fig:coolstream}
has been selected by eye and is not physically motivated (e.g., by a predicted orbit)
but it can be observed as a coherent-appearing stream in the \thirteenco data cubes
as well.

This cool stream projected along the line of sight to the CMZ may mitigate the
star formation deficit noted in \citet{Longmore2013b}.  The cool stream does
not appear to be actively star forming at present, most likely because the gas
is at a lower average density, yet it contributes to the total cold gas mass
inferred in the inner galaxy.  However, the total mass within the inner
projected 200 pc is still dominated by the `ring' clouds and Sgr B2, so at
least some deficit remains.


\RotFigureTwoAA
{figures/orbits/mydrawnpathpv_on_H2CO_DendrogramTemperature_smooth.pdf}
{figures/orbits/mydrawnpath_on_H2CO_DendrogramTemperature_smooth.pdf}
{Position-velocity slice tracing the ``fresh stream'' of cooler molecular
gas from Section \ref{sec:coolstream}.}
{fig:coolstream}{1}{9.5in}

\subsection{Is the gas warm or just optically thick?}
\label{sec:thickorwarm}
As noted in Section \ref{sec:h2cocompare}, there is degeneracy in the modeling
that allows the observed ratios to be produced by very dense gas in addition to
hot gas.  Indeed, ratios approaching $\Rone=1$ occur for relatively low
temperatures ($T\sim30$ K) at very high densities ($n>10^{6}$ \percc)
and very high column densities ($N(\para)>10^{16}$ \perkms \persc) because
the lines become optically thick and have LTE temperatures.

Since the mean densities at 30\arcsec resolution are always significantly lower
than this value ($n\lesssim\mathrm{few}\ee{5}$ \percc), such high density,
optically thick gas would have to have a very low filling factor.  The peak
\para \threeohthree brightness ranged from $0.2 \lesssim T_{mb} \lesssim 2$ K
in the detected regions, implying upper limits on the filling factor 1-7\%.
In practice, the filling factor would have to be lower still, since any optically
thin \para would favor the \threeohthree line and contribute to a lower \Rone.

The strongest argument against such a scenario is the extremely high column
density required.  At typical \para abundances $X\sim10^9$, or even more
extreme abundances $X\sim10^8$, the implied local column density is
$N(\hh)\gtrsim10^{25(24)}$ in a single velocity bin.  Column densities up to
$N(\hh)\sim10^{24}$ \percc are observed, but with much lower filling fractions
$ff\sim10^{-4}$, in The Brick \citep[][their Figure 4]{Rathborne2014a}.  Even
in The Brick, the densest cloud besides Sgr B2, higher column densities are not
observed.  So, while it remains possible that optically thick \para is
responsible for the observed line ratios, it is very unlikely, especially
outside the densest clouds.

\subsection{The CMZ average}
For comparison to extragalactic observations, we have included a spectrum
averaged over the whole $\sim300$ pc extent of our survey in Figure
\ref{fig:wholecmzspec}.  The peak amplitude is 50-60 mK and the measured
line ratio $\Rone = 0.25$, corresponding to a temperature $T=60$ K.  
Over the same area, the measured mean dust temperature from HiGal is 23 K,
though on such large scales it would be more appropriate to use Planck or WMAP
data to perform this measurement.
The spectral fit is imperfect, showing a non-flat baseline between the
\threetwoone and \threetwotwo lines, but nonetheless the extracted value is
likely representative of the CMZ-wide average.  This temperature is very close
to the \ammonia-measured temperature in M83 \citep[56 K;][]{Mangum2013a}, which
also has a similarly low dust temperature $T_{dust,M83} = 31$ K over a 600 pc
region.  Maffei 2, NGC 1365, and NGC 6946 also exhibit similar gas and dust
temperatures to our Galactic center in the \citet{Mangum2013a} sample,
suggesting that these galaxies all have analogous gas thermal structures in
their inner regions.  The other galaxies in that sample, especially the
starburst galaxies, have significantly warmer gas and dust, with temperatures
closer to those measured for Sgr B2 than for the whole CMZ.


\Figure
{figures/simple/WholeCMZ_6parameter.pdf}
{The spectrum of the whole survey averaged over all pixels.
The fit parameters, along with the nominal errors on the parameters,
are shown in the legend.  A single-component fit was used, though many
subtler individual components are evident.}
{fig:wholecmzspec}{0.5}{0}

% However, it is still possible
% to distinguish cosmic ray and turbulent heating if one dominates over the
% other: turbulent heating depends strongly on the velocity dispersion ($\Gamma
% \propto \sigma_V^3$), while CR heating should have no dependence on the
% velocity dispersion.  Additionally, turbulence has a scale dependence, since
% the global dynamics (rotation) dominates over turbulence on scales larger than
% the scale height in the CMZ: $h\sim10$ pc (4\arcmin) in the 100-pc `ring',
% $h\sim50$ pc (20\arcmin) in the $\ell=1.3\arcdeg$ cloud \citep{Kruijssen2013a}.
% I'm not sure this really makes sense.  Is it even possible to measure the gas
% temperature on larger physical/angular scales?

% These differences result in testable predictions.  If turbulence is the
% dominant heating mechanism in the CMZ clouds, there should be a correlation
% between the observed line width and the gas kinetic temperature.

% \todo{This is speculative.}  In a highly turbulent medium, most of the power
% dissipation happens on the smallest size scale (if $L > \sigma_v^3$).  If the
% cooling rate is enhanced in the postshock medium, the high gas temperature will
% be isolated to high-density, compact clumps, but these clumps will also
% dominate the line emission.  \todo{This should result in a density dependence
% of the temperature or observed temperature $T(n) > n^1$, maybe $T(n) \sim n^2$ ?}
% Cosmic rays should not exhibit this behavior; they should heat all scales equally
% and care only about the \emph{mean} density.

%\subsection{Shocks}
%Assuming the dust temperature reflects the mean gas temperature and only a subset
%of the gas has been heated to the high observed temperatures, we can determine the
%shock velocity required to supply the heating.  Assuming an adiabatic strong shock,
%$T_{ps} = \frac{3}{16}\frac{\mu v_s^2}{k}$, where $\mu = \mu_{\hh} = 2.8$.  A 1 \kms
%strong shock can produce a postshock temperature of 65 K.  With observed linewidths 

%\section{Examination of Individual Spectra}
%%\todo{The modeling is ambiguous still: it is necessary to impose some sort of
%%restriction on the abundance or column density in order to acquire reasonable
%%constraints on the temperature.}
%In order to maximize the signal-to-noise and acquire the strongest constraint
%on the temperature, we extract spectra for regions with bright lines.  This
%approach is similar to the approach adopted in \citet{Ao2013a}, but for a much
%larger area.
%
%\subsection{Sources in the Ring}
%
%\subsubsection{The Brick: G0.253+0.016}
%We examine two lines of sight through The Brick, in the northeast at
%G0.241+0.006 and in the southwest at G0.261+0.028.  The northeast line of sight
%has two independent velocity components with dramatically different
%temperatures: a 30 \kms component with only a lower limit on temperature,
%$T>300$ K, and a 0 \kms component with $T=73\pm3$ K.  The southwest line of
%sight has only one component with $T=157\pm10$ K.
%
%\subsubsection{Cloud d: G0.38+0.04}
%Cloud ``d'' is one of the possible protoclusters identified in
%\citet{Longmore2013a}.  It exhibits bright and relatively narrow (FWHM$\sim7.3$
%\kms) \para emission and has a tightly constrained temperature $T=84\pm5$ K.
%
%\Figure{figures/G0.38+0.04_fit_h2co_mm_radex.pdf}
%{The 218-219 GHz spectrum of Cloud d}
%{fig:cloudcspec}{0.5}{0.0}
%
%\subsection{Clouds e/f: G0.47+0.01}
%Clouds ``e'' and ``f'' together make another proto-cluster in
%\citet{Longmore2013a}.  These clouds are marginally warmer than cloud d, with
%$T=110\pm20$ K.  However, because of the region of parameter space the lines
%allow, the density is reasonably constrained to be $n(\hh)\sim10^{4\pm0.2}$ \percc,
%much lower than in cloud d and more comparable to The Brick.

\section{Conclusion}
We present the largest gas temperature map of the CMZ to date.
There is warm ($T\gtrsim60$ K) pervading the CMZ out to a radius $\sim200$ pc.
However, there is also significant contrast in gas temperature, with massive
clouds such as The Brick, Sgr B2, and the Sgr A cloud complex exhibiting
temperatures in excess of $T>100$ K.

Our new observations are broadly consistent with the orbital model described by
\citet{Kruijssen2014d}.  While there are some individual discrepancies, it at
least appears that the front-side clouds (The Brick and the lettered clouds on
the path to Sgr B2) have consistent thermal properties and fall along a stream
of progressively increasing temperature as a function of age.  However, The
Brick itself is an outlier in this progression.

The 80 \kms cloud stream extending from $\ell=1.2$ to $\ell=0.3$ is relatively
cool compared to the rest of the CMZ despite its broad line width.  The lower
temperature hints that it is more distant from pericenter than the `ring'
clouds, and it may therefore represent a fresh stream of gas infalling into the
center.

\textbf{Acknowledgements}:
We thank the staff and observers at APEX for carrying out the service-mode
observations.  We are grateful to Arnaud Belloche, Axel Wieß, and Carlos de
Breuck for assistance in developing the observing strategy, and Per Bergman for
his assistance in understanding the SHFI-1 baseline issues.
\todo{Please provide any acknowledgements that are needed here.}

\textbf{Code Packages Used}:

\begin{itemize}
    \item sdpy \url{https://github.com/adamginsburg/sdpy}
    \item aplpy \url{http://aplpy.github.io}
    \item pyradex \url{https://github.com/adamginsburg/pyradex}
    \item myRadex \url{https://github.com/fjdu/myRadex}
    \item pyspeckit \url{http://pyspeckit.bitbucket.org}
    \item aplpy \url{https://aplpy.github.io/}
    \item wcsaxes \url{http://wcsaxes.readthedocs.org}
    \item spectral cube \url{http://spectral-cube.readthedocs.org}
    \item pvextractor \url{http://pvextractor.readthedocs.org/}
\end{itemize}

\appendix
\section{Baseline Removal}
\label{sec:baselineappendix}
The baselines in our SHFI-1 data were particularly problematic, more than is
usual in modern heterodyne obervations.  The 218 GHz window we have observed is
particularly sensitive to resonances within the SHFI-1 receiver that vary on
$<1$ minute timescales; our off-position calibrations were performed about once
per minute and therefore were not rapid enough to mitigate this problem
completely.  The baselines can broadly be described as smoothly varying ripples
on the scale of $\sim1/10$ the spectral window, plus more rapidly varying
ripples on 20-40 \kms scales.  In principle, this is a straightforward problem
of identifying the fourier components associated with each of these scales and
subtracting them.

In practice, we discovered that it was not possible to remove the dominant
baseline structure on either scale without significantly affecting the
underlying spectral data.  We simulated a variety of fourier-space suppression
approaches by adding synthetic signal to baseline spectra extracted from the
first few PCA components of the real spectra.
The PCA extraction approach is able to pull out the dominant baseline
components very effectively, but it inevitably includes significant signal in
the top few most correlated components, especially for the strong \formaldehyde
and \thirteenco lines.  We therefore abandoned it for the final data reduction.

For reference, we show an example spectrum that we believe to consist entirely
of baseline ripples in Figure \ref{fig:badbaselines}.

\Figure{figures/worst_baselines_map001_withsynth}
{An example showing some of the worst baselines observed.  The plotted spectrum
is from an observation on April 2, 2014, showing the average of the 5\% worst
spectra.  The blue curve shows a $\sigma=5$ \kms (FWHM$=11.75$ \kms) line centered
at the 0 \kms position of \para \threeohthree,
illustrating that the baseline `ripples' have widths comparable to the observed
lines.  While we selected the worst 5\% in this case, nearly all spectra are
affected by these sorts of baselines, and the shape and amplitude varies
dramatically and unpredictably.  The variation, unpredictable though it is,
works in our favor as it averages out over multiple independent observations.
The 218 GHz region shown here is also the worst-affected; the 220 GHz range
that includes the \thirteenco lines generally exhibits smoother and
lower-amplitude baseline spectra.}
{fig:badbaselines}{0.6}{0}

%\Figure{figures/M-091.F-0019-2013-2013-06-08_PCA_high_diagnostic.png}
%{The first 3 eigenspectra from an observation taken on June 8, 2013.  These
%show the baselines that are removed by PCA cleaning.  They do not include any
%signal from the spectral lines.  \todo{This will be
%replaced with a higher-quality figure}}
%{fig:eigenspectra}{0.5}{0}

\section{Ratio Examination of Individual Regions}
\todo{While figures have been made for this, they will not be included unless
necessary.}
To further examine the reliability of the ratio determination, we have examined
individual regions in the unsmoothed images on a pixel-by-pixel basis, using
\texttt{pyspeckit} to fit a 4-parameter model (amplitude, centroid, width, and
$R_1$) to each spectrum with constrained velocity range and width.  These fits
were performed on the locally baseline-subtracted data cubes.

\section{Spectral fits for regions and apertures}
\todo{This section is intended to be electronic-only.}
We provide a catalog of line ratios measured from aperture-extracted spectra
along with figures showing the best fit model for each spectrum.  We have
extracted spectra in regions with significant \para emission and in commonly
studied individual regions.  This catalog is intended to provide archival value
from our data set and enable future studies of individual regions.

In the attached electronic table, three types of region are included: 
\begin{enumerate}
    \item Circular regions-of-interest (Figure \ref{fig:regions}a)
    \item Rectangular regions-of-interest (Figure \ref{fig:regions}b)
    \item $8\arcmin \times 8\arcmin$ square observation fields (Figure \ref{fig:regions}c)
\end{enumerate}
The fit tables include multicomponent spectral fits with and without
spline-based baseline removal (see Section \ref{sec:linemodeling}).
Spatial parameters for the box (\texttt{GLON}, \texttt{GLAT},
\texttt{boxheight}, and \texttt{boxwidth}) or circle (\texttt{GLON},
\texttt{GLAT}, and \texttt{radius}) regions are included.  The average dust
column and dust temperature are reported (\texttt{higaldusttem} and
\texttt{higalcolumndens}).  The best fit parameters for gas temperature, \para
column, and \hh density are included along with their $\pm1-sigma$ marginalized
limits.

\RotFigureThreeAA
{figures/regions/boxes_on_h2co}
{figures/regions/circles_on_h2co}
{figures/regions/square_fields_on_h2co}
{Figures showing the extracted regions.  The background is the integrated
masked \para \threeohthree image.}
{fig:regions}{1}{9.5in}

\section{Source code and Data release}
The reduced and raw data are made available via the CfA dataverse
(doi:10.7910/DVN/27601).

The source code for this project in its entirety is available in the attached
tar file and on the internet at
\url{https://github.com/adamginsburg/APEX_CMZ_H2CO}.
Because the archives are public, we include scripts to download and process the
raw data so that all steps of the analysis performed here can be performed by
any individual with access to a computer.  Most of the data sets can be retrieved
in an automated fashion using an \texttt{astroquery}-based script from the ESO archive,
but some of the data is not stored in the archive and must be retrieved manually from the
\texttt{dataverse} servers. The total data reduction process
takes $\sim30$ hours on a 48-core, 2012-era linux machine, though most steps
are not parallelized so the process may be faster on more recent machines with
fewer cores.  About 300 GB of free space are required to store the raw,
intermediate, and reduced data products, though the final products are $<30$
GB, with an actual size depending on whether the baseline-subtracted versions
are kept separate from the original files.  The download process may take
substantially longer than the data reduction.



\ifstandalone
\bibliographystyle{apj_w_etal}  % or "siam", or "alpha", or "abbrv"
\bibliography{bibdesk}      % bib database file refs.bib
\fi

\end{document}

