\input{preface}

\title{APEX CMZ Survey}
\newcommand{\eso}     {$^{1 }$}
\newcommand{\mpifr}   {$^{2 }$}
\newcommand{\saudi}   {$^{3 }$}
\newcommand{\naoj}    {$^{4 }$}
\newcommand{\pmo}     {$^{5 }$}
\newcommand{\nrao}    {$^{6 }$}
\newcommand{\casa}    {$^{7 }$}
\newcommand{\cfa}     {$^{8 }$}
\newcommand{\chalmers}{$^{9 }$}
\newcommand{\oxford}  {$^{10}$}
\newcommand{\mpa}     {$^{11}$}
\newcommand{\ljmu}    {$^{12}$}
\newcommand{\lmu}     {$^{13}$}

\author{Adam Ginsburg{\eso},
        Christian Henkel{\mpifr,\saudi},
        Yiping Ao{\naoj,\pmo},
        Denise Riquelme{\mpifr},
        Jens Kauffmann{\mpifr},
        Thushara Pillai{\mpifr},
        Elizabeth A.C. Mills{\nrao},
        M. A. Requena-Torres{\mpifr},
        Katharina Immer{\eso},
        Leonardo Testi{\eso},
        Juergen Ott{\nrao},
        John Bally{\casa},
        Cara Battersby{\cfa},
        Jeremy Darling{\casa},
        Susanne Aalto{\chalmers},
        Thomas Stanke{\eso},
        Sarah Kendrew{\oxford},
        J.M. Diederik Kruijssen{\mpa},
        Steven Longmore{\ljmu},
        James Dale{\lmu},
        %Alexandre Faure{}, % invite him first
        Rolf Guesten{\mpifr},
        K.M. Menten{\mpifr}
        }

%\institute{
%      {$^\casa$}{\it{CASA, University of Colorado, 389-UCB, Boulder, CO 80309}}}
%      {$^\eso$}{\it{European Southern Observatory, Karl-Schwarzschild-Strasse 2, D-85748 Garching bei München, Germany}}}
%      {$^\cfa$}{\it{CfA}}}
%      {$^\mpifr$}{\it{Max Planck Institute for Radio Astronomy, auf dem Hugel, Bonn}}}
%      {$^\nrao$}{\it{National Radio Astronomy Observatory, Socorro}}}
%      {$^\oxford$}{\it{Oxford}}}
%      {$^\chalmers$}{\it{Chalmers}}}
%}
\institute{
    {\eso}{\it{European Southern Observatory, Karl-Schwarzschild-Strasse 2, D-85748 Garching bei München, Germany\\
                      \email{Adam.Ginsburg@eso.org}}} \\ 
    {\mpifr}{\it{Max Planck Institute for Radio Astronomy, auf dem Hugel, Bonn}}
    {\saudi}{\it{Astron. Dept., King Abdulaziz University, P.O. Box 80203,
    Jeddah 21589, Saudi Arabia}}\\
    {\naoj}{National Astronomical Observatory of Japan, 2-21-1 Osawa, Mitaka, Tokyo 181-8588, Japan}
    {\pmo}{Purple Mountain Observatory, Chinese Academy of Sciences, Nanjing 210008, China}
    {\nrao}{\it{National Radio Astronomy Observatory, Socorro}}
    {\casa}{\it{CASA, University of Colorado, 389-UCB, Boulder, CO 80309}} \\ 
    {\cfa}{\it{Harvard-Smithsonian Center for Astrophysics, 60 Garden
    Street, Cambridge, MA 02138, USA}} \\ 
    {\chalmers}{\it{Department of Earth and Space Sciences,
                    Chalmers University of Technology}}
    {\oxford}{\it{Department of Astrophysics, The Denys Wilkinson Building, Keble Road, Oxford OX1 3RH}}
    %{\edmonton}{\it{University of Alberta, Department of Physics, 4-181 CCIS, Edmonton AB T6G 2E1 Canada}} \\ 
    %{\yale}{\it{Department of Astronomy, Yale University, P.O. Box 208101, New Haven, CT 06520-8101 USA}} \\ 
    %{\puertorico}{\it{Department of Physical Sciences, University of Puerto Rico, P.O. Box 23323, San Juan, PR 00931}}
    {\mpa}{\it{Max-Planck Institut f\"{u}r Astrophysik, Karl-Schwarzschild-Stra\ss e 1, 85748 Garching, Germany}}
    {\ljmu}{\it{Astrophysics Research Institute, Liverpool John Moores
    University, IC2, Liverpool Science Park, 146 Brownlow Hill, Liverpool L3
    5RF, United Kingdom}}
    {\lmu}{\it{University Observatory Munich, Scheinerstr. 1, D-81679 München, Germany}}
    }


\section{Abstract}

\footnote{Compiled on \today\ at \currenttime}


\section{Introduction}
\todo{To do.  To-do items are coded in red.}




\section{Observations \& Data Reduction}

\subsection{Observations}
We observed the Central Molecular Zone from $-0.4 < \ell < 1.6$ with the SHFI-1
instrument \citep{Vassilev2008a} on the APEX telescope using the XFFTS backend.
The observations were divided into 25 hours in June 2013 and 75 hours in
April-July 2014.  

The 2013 observations were taken in $4\arcmin \times 4\arcmin$ patches, and the
frequency range covered was 217.5-220 GHz and 216-218.5 GHz in the two spectral
windows.  Scans were performed along lines of constant RA and Dec.

The 2014 observing strategy was modified to use larger $8\arcmin \times
8\arcmin$ scans along lines of Galactic latitude and longitude.  The frequency
range was also shifted to cover windows over 217-219.5 and 218.4-220.9 GHz, thus
including the bright \thirteenco and \ceighteeno 2-1 lines.

The raw data were acquired with 32768 spectral channels in each window, yielding
0.1 \kms resolution.  We downsampled the data by a factor of 8 to 0.8 \kms
resolution to make the data more manageable.  We also expect to see no lines
narrower than a few \kms in the CMZ, particularly not with the relatively
shallow observations we have acquired.

\subsection{Reduction}
\subsubsection{Calibration}
Calibration was performed at the telescope using the standard APEX calibration
tools
\footnote{See the observing manuals:
\url{www.apex-telescope.org/documents/public/APEX-MPI-MAN-0012.pdf}
\url{www.apex-telescope.org/documents/public/APEX-MPI-MAN-0013.pdf} }.
These yield flux-calibrated spectra at each position with appropriate pointing
information.  Typical flux calibration uncertainties are $\sim$ \todo{???} and
pointing errors \todo{???} ($\sigma<2$\arcsec ?).

There was a significant calibration error discovered at the APEX telescope during
a large segment of the 2014 observing campaign
(\url{http://www.apex-telescope.org/heterodyne/shfi/calibration/calfactor/}).
The calibration errors were of order $\sim15-25\%$.  They were corrected after
the fact, but the remaining calibration uncertainty
is higher in this data, $\sim15\%$ total rather than the usual $\sim10\%$.

\subsubsection{Flagging bad spectra}
Spectra were removed if they showed excessive noise compared to the
theoretically expectation given the measured system temperature.  As in
\citet{Ao2013a}, the threshold was set to $1.5\times$ the theoretical noise,
i.e. $1.5 \sqrt{2} T_{sys} (\Delta\nu t_{exp})^0.5$, where $t_{exp}$ is the
exposure time (integration time) per spectrum in seconds and $\Delta\nu$ is the
channel width in Hz.  This approach resulted in $\sim0.2\%$ of the data being
removed.  

At the position of Sgr B2, the noise was significantly higher due to signal
from the continuum source.  We therefore disabled flagging in a
$\sim2.5\arcmin$ box around Sgr B2.


% Removing these baselines by
% the usual technique of masking out the emitting spectral region and fitting a
% low-order polynomial to the surroundings is somewhat impractical in the
% Galactic Center, where baseline ripples are on comparable scales to the
% extremely wide spectral lines.

%This method was abandoned: it resulted in negative bowls
% We perform a scan-by-scan baseline subtraction as follows: 
% \begin{enumerate}
%     \item Compute the mean spectrum across the scan
%     \item Identify pixels $>1-\sigma$ above the mean spectrum to be ignored
%         when baseline fitting (this is a simple `line masking' process)
%     \item Fit a line to each spectral channel across the scan
%     \item Smooth the fitted line parameters with a gaussian with with
%         $\sigma=10$ channels, interpolating into the line-masked regions
%     \item Subtract the fitted baselines
% \end{enumerate}
% 
% We therefore adopted a principle component analysis (PCA) baseline removal
% approach in which the data were spectrally smoothed and downsampled by an
% additional factor of 25, for a characteristic frequency scale of 15 MHz.  The
% data had their means subtracted before PCA cleaning so that only time-variable
% baseline variation was removed.  The resulting data were decomposed into
% eigenspectra, of which the 3 most correlated components were then set to zero.
% These components visually match the baseline ripples, and due to the wide
% smoothing (20 \kms resolution), no line information is lost.  The selection of
% 3 components was determined empirically by examining the eigenspectra; a test
% removing the first 10 components showed no significant difference in the
% cleaning but did show signs of extracting large-scale correlated signal.
% 
% \Figure{figures/M-091.F-0019-2013-2013-06-08_PCA_high_diagnostic.png}
% {The first 3 eigenspectra from an observation taken on June 8, 2013.  These
% show the baselines that are removed by PCA cleaning.  They do not include any
% signal from the spectral lines.  \todo{This will be
% replaced with a higher-quality figure}}
% {fig:eigenspectra}{0.5}{0}


\subsubsection{Mapmaking}
The maps were made by computing an output grid in Galactic coordinates with
7.2\arcsec pixels and adding each spectrum to the appropriate pixel\footnote{We
use the term `pixel' to refer to a square data element projected on the sky
with axes in Galactic coordinates.  The term `voxel' is used to indicate a cubic data
element, with two axes in galactic coordinates and a third in frequency or
velocity}.  In order
to avoid empty pixels and maximize the signal-to-noise, the spectra were added
to the grid with a weight set from a gaussian with $FWHM=10\arcsec$, which
effectively smooths the output images from $FWHM\approx28\arcsec$ to
$\approx30\arcsec$.  See \citet{Mangum2007a} for more detail on the on-the-fly
mapping technique used here.

%\todo{To demonstrate the utility of the PCA subtraction approach described in Section
%\ref{sec:baseline}, we show maps of the \formaldehyde \threeohthree line before
%and after PCA subtraction.}

The PPV cubes were created with units of brightness temperature ($T_A^*$).  The
main beam efficiency is $\eta_{mb} = 0.75$ (gain $\sim39$
Jy/K)\footnote{\url{http://www.apex-telescope.org/telescope/efficiency/}}.



\subsection{Baselining}
\label{sec:baseline}
The data showed significant baseline structure, leading to large-scale
correlated components in the resulting spectra.  The baselines were removed
after cube generation by first identifying bright regions in the \para
\threeohthree line using the technique described in Section \ref{sec:signal},
masking those out for each other line, then fitting a 5th-order polynomial to
the remaining spectra over the velocity range -150 to 250 \kms and subtracting
it.

After performing this baselining, the signal was re-extracted.

\section{Ratio and Temperature Maps}
\label{sec:h2co}
We created two independent ratio maps, \threeohthree/\threetwotwo and
\threeohthree/\threetwoone, so that the two ratios (which are expected to be
the same) can be used as a sanity check.  However, in many regions, especially
those with the greatest signal, the \threetwotwo line is strongly affected by
contamination from a \methanol line, so the ratio is not particularly reliable.

Ratio maps are subject to dramatic uncertainties in the low signal-to-noise
regime, which is inevitably reached in some parts of the map.  The error
distribution on the ratio becomes non-Gaussian, approaching a pathological
Cauchy distribution in which the mean becomes undefined as the mean of the
individual values being compared approaches zero.  We therefore sought other
means to create temperature maps.

\section{Signal Extraction \& Analysis}
\label{sec:signal}
We use the method described partially in \citet{Ao2013a} and more thoroughly in
\citet{Dame2011b} to mask the data cubes at locations of significant signal in 
the brightest line. 
A noise map was created by computing the sample standard deviation over a
200-\kms range in which no signal was present.
We use the \formaldehyde \threeohthree line to create the mask by
the following procedure:

\begin{enumerate}
    \item Smooth the data with a gaussian of width 2 pixels in each direction
        (spatial and spectral)
    \item Identify all pixels with brightness $T_A > 3\sigma$ in the smoothed
        map
    \item Grow the mask from the previous step by 1 pixel in each direction
        (this is known in image processing as binary dilation)
\end{enumerate}

The \formaldehyde \threeohthree mask was then applied to the \threetwoone and
\threetwotwo cubes created with the same PPV gridding.  There is some overlap
between the \methanol \fourtwotwo line and the \formaldehyde \threetwotwo
line in PPV space, so we shifted the \formaldehyde mask to the velocity of the
\methanol line in the \formaldehyde \threetwotwo cube and excluded any pixels
expected to have signal.

Figure \ref{fig:ratiomaps} show the ratio of the integrated emission in the
masked regions.  The \threetwotwo line is missing in large sections of the map
due to \methanol contamination, and therefore is not very useful in constraining
the gas temperature, so it is not included.


% figure_ratio_maps
\Figure
%{figures/big_H2CO_322221_to_303202_smooth_bl_integ.pdf}
{figures/big_H2CO_321220_to_303202_smooth_bl_integ.pdf}
{Maps of the ratio of the integrated emission ratio
\threetwoone/\threeohthree (bottom).  The data cubes were masked by
signal-to-noise ratio in the \threeohthree line, with blank (gray) regions
indicating nondetections.  The Sgr B2 peaks exhibit \formaldehyde
absorption and therefore are not reliable.  Higher ratios (red) correspond
to higher temperatures.
}
%{fig:ratiomaps}{1}{7in}
{fig:ratiomaps}{0.6}{0}

\subsection{Region Extraction}
In order to extract higher signal-to-noise measurements on selected regions, we
broke the data down into subsets using both a by-eye region selection drawing
regions using ds9 and a more systematic approach using a dendrogram-based clump
finding algorithm \citep[][\url{http://dendrograms.org/}]{Rosolowsky2008c}.

We ran the dendrogram/clumpfind extraction analysis on the full resolution
\threeohthree cube data.  We enforced thresholds of a minimum number of pixels
$N_{min}=100$, a peak threshold of $I = 3\sigma$, and a splitting threshold
$\Delta = 2\sigma$.  The exact values of these parameters is not particularly
important, as we are interested in general trends with size-scale and
galactocentric distance, but we caution against overinterpretation of the
resulting catalog as the number of sources and their size and distribution can
change dramatically with small changes in the selected parameters.

For each extracted blob, we measured the corresponding integrated and peak
emission in the \threetwoone and \thirteenco cubes.  We also extracted the mean
dust temperature and column density from SED fits to Herschel HiGal 170-500\um
maps.  The SED fits were performed on background-subtracted data using an
approach originally described in \citet{Battersby2011a}, which includes details
of the fitting process and specification of assumed physical parameters.  We
also extracted dust column density and temperature using a more naive
pixel-by-pixel approach with no background subtraction
(\url{http://hi-gal-sed-fitter.readthedocs.org/en/latest/}) and found the two
to be consistent.

\subsection{Line Modeling}
\label{sec:linemodeling}
\todo{Different grids have been used at different stages.  The specific numbers
will need to be updated.}
We use RADEX \citet{van-Der-Tak2007a} to create a model grid for the
p-\formaldehyde lines over a sparse grid of density ($n=10^{4,5,6,7}$ \percc),
a fixed assumed line gradient of 5 \kms / pc, a fixed column of \formaldehyde
$N(\formaldehyde) = 9.5\ee{13}$ \persc (which corresponds to an abundance
$X=10^{-8.5}$ at $N(\hh) = 3\ee{22}$ \persc, and a range of temperatures from
10 to 300 K with 100 evenly spaced grid points.  The grid was created using
the \texttt{pyradex} wrapper of
RADEX\footnote{\url{https://github.com/keflavich/pyradex}}.
We then upsampled these grids by spline interpolating between grid points to
acquire a high-resolution ($T\times n \times N = 250\times100\times100$) grid
covering $10<T<350$ K, $10^{2.5} < n < 10^7$ \percc, and $10^{11} < N <
10^{15.1}$ \persc\perkmspc.

We extracted spectra averaged over each `by-eye' selected region and fitted
a 6-parameter model to the full 218-219 GHz spectral range.  The fitted
parameters are the amplitude of the \formaldehyde \threeohthree line, the
centroid velocity, the line width ($\sigma$, not FWHM), the ratio $R_1 =
\Rone$, the ratio $R_2 = \Rtwo$, and the amplitude of the \methanol \fourtwotwo
line.


For each extracted region, we created $\chi^2$ grids using independent
constraints from the line ratio, the \formaldehyde abundance, and the total
column density of \hh.  We use the line \emph{ratio} rather than line
brightnesses to avoid uncertainties due to the ``filling factor'' of the
emitting gas: even for diffuse clouds, the filling factor of the emitting
regions may be $ff<<1$ if the emission is isolated to compact shocked regions,
as expected if highly supersonic $\mathcal{M}>10$ turbulence is energetically
dominant in the clouds.  We use an abundance $N(\para)/N(\hh) = X(\para) =
10^{9.08\pm1} = 1.2\ee{9} \times/\div 10$, allowing for dramatic uncertainty in
the \formaldehyde abundance
\citep{Ginsburg2013a,Carey1998a,Wootten1978a,Mundy1987a}.  To constrain the
total column density, we use the Herschel dust maps to derive an \hh column
density, which has a nominal $\sim2-3\times$ uncertainty.  We treat the error
as $10\times$ to account for the abundance uncertainty.  This large assumed
error also accounts for factor $\sim2-3$ uncertainty due to line-of-sight
confusion, since the dust column density cannot be associated with any specific
velocity component.

These constraints are shown projected onto the three planes of our fitted
parameter space in the multi-paneled Figures
\ref{fig:coltemconstraints}-\ref{fig:parsonbrightness}.  The fitted parameters
are displayed in Figure \ref{fig:parsonbrightness}.

While $R_2$ can, in some cases, provide significant constraints on various
parameters, we do not use it because of the ambiguity imposed by the overlap
with the \methanol \fourtwotwo line.

\Figure{figures/simple/d:G0.38+0.04_fit_4_lines_simple.pdf}
{Fitted spectrum of ``cloud d".  The fitted parameters and their corresponding
errors are shown in the legend.  The parameters are the amplitude of the
\formaldehyde \threeohthree line, the centroid velocity, the line width
($\sigma$, not FWHM), the ratio $R_1 = \Rone$,
the ratio $R_2 = \Rtwo$, and the amplitude of
the \methanol \fourtwotwo line.  }
{fig:clouddspec}{0.5}{0}

\Figure{figures/param_fits/d:G0.38+0.04_col_tem_0_parameter_constraints.pdf}
{The parameter constraints for ``cloud d'' (Figure \ref{fig:clouddspec})
projected (marginalized) onto the temperature/column density plane.\newline
(top left) Constraints imposed by the measured ratio \Rone are shown in filled contours,
with blue corresponding to $\Delta\chi^2 < 1$, cyan $1 < \Delta\chi^2 < 2$,
yellow $2 < \Delta\chi^2 < 3$, and red $3 < \Delta\chi^2 < 4$.  The solid
contours show the joint constraints imposed by including restrictions on the
total column density, filling factor, and abundance, following the same color
scheme.  The solid grayscale contours show the constraints imposed by
$R_2=\Rtwo$, from black ($\Delta\chi^2 < 1$) to white.  In this case, $R_2$
imposes only a weak lower limit on column. \newline
(top right) The same colorscheme as before, showing the constraints imposed by
assuming the abundance of \para relative to \hh is as labeled.  In this case,
the abundance does not constraint these parameters. \newline
(bottom left) The same colorscheme as before, showing the constraints imposed
by the measured total column density of \hh, converted to a constraint on the
\para column by assuming a velocity gradient (5 \kms \perpc) and abundance as
shown in the top right panel.  The uncertainty is dominated by the abundance
uncertainty.  \newline
(bottom right) The same colorscheme as before, showing the constraints imposed
by forcing the filling factor of the line emission to be $ff < 1$.
}
{fig:coltemconstraints}{0.5}{0}

\Figure{figures/param_fits/d:G0.38+0.04_dens_col_0_parameter_constraints.pdf}
{The constraints in density-column parameter space.
See Figure \ref{fig:coltemconstraints} for details.  In the bottom-right panel, the
solid line shows the contour corresponding to $ff(\threeohthree)=0.1$ in
black.}
{fig:denscolconstraints}{0.5}{0}

\Figure{figures/param_fits/d:G0.38+0.04_dens_tem_0_parameter_constraints.pdf}
{The constraints in density-temperature parameter space.
See Figure \ref{fig:coltemconstraints} for details.}
{fig:denstemconstraints}{0.5}{0}

\Figure{figures/param_fits/d:G0.38+0.04_0_h2coratio_minaxis.pdf}
{The line brightness of \para \threeohthree (top row) and \para \threetwoone
(bottom row) in the three different projections of parameter space.  The
grayscale images correspond to a slice through the parameter spaces at the
location of the best-fit parameter.  The colored contours show the allowed
marginalized regions in each parameter space as described in the
Figure \ref{fig:coltemconstraints} caption.}
{fig:parsonbrightness}{0.5}{0}

\subsection{Blob modeling}
\label{sec:dendromod}
Similar to the line modeling approach, we derived temperatures for each
dendrogram catalog blob using the measured line ratio and dust column density.
The $\chi^2$ grid approach described in Section \ref{sec:linemodeling} was used
to extract the temperature and associated error.

We plot the derived temperature for all clumps against the measured line ratio
in Figure \ref{fig:ratiovstem}.  The relation is nearly monotonic for the vast
majority of sources.  To simplify further analysis, we fit a three-element
piecewise linear model for temperature as a function of $R_1$.  We use this
function for further derivations of the temperature.  There is noticeable
scatter above $R_1 > 0.4$ and $T>150$ K, but in this regime the modeling is
already intrinsically uncertain and these ratios are best treated as lower
limits on the gas temperature.

% dendrotem_plots
\Figure{figures/dendrotem/ratio_vs_temperature_piecewise.pdf}
{The derived temperature vs the measured ratio \Rone.  The temperature
includes constraints from the assumed fixed \formaldehyde abundance and the 
varying column density.  The points are color coded by signal-to-noise in the
ratio $R_1$, with black $S/N < 5$, blue $5 < S/N < 25$, green $25 < S/N < 50$,
and red $S/N > 50$.  The black line shows a three-element piecewise fit
to the data.
}
{fig:ratiovstem}{0.6}{0}

\subsection{Temperature Maps}
\label{sec:formaldehydetemmap}
Using the piecewise relation derived in Section \ref{sec:dendromod}, we
converted the ratio map shown in Figure \ref{fig:ratiomaps} to a temperature
map in Figure \ref{fig:temmap}.  Because this map includes the full line-of-sight
integrated emission, there are many regions where multiple independent components
are being mixed.

Figure \ref{fig:dendrotemmap} shows a temperature map created by measuring the
temperature in each clump from the dendrogram clump extraction based on the
smoothed data cube and averaging over the velocity axis.  Because there are
hierarchically nested clumps in the dendrogram-based clumpfind, and because
there are genuinely independent clumps along common lines of sight, there is
still overlap and confusion.  The nested clumps in the dendrogram can provide
more than one temperature measurement per voxel; we therefore use the most
compact clump's temperature measurement for each voxel.

The two temperature maps broadly agree, but the dendrogram map is smoother and
aesthetically superior.  The noisy edges of regions in Figure \ref{fig:temmap}
are unlikely to be real temperature features, so the dendrogram map is
preferred.

\Figure{figures/big_lores_smooth_tmap_withcontours.pdf}
{A temperature map derived from the \Rone ratio map shown
in Figure \ref{fig:ratiomaps}.  Regions of lower signal-to-noise, and therefore less
reliable temperature, are grayed out with a filter that gets more opaque
toward lower signal-to-noise.  Temperatures above $\gtrsim150$ K should be
treated as lower limits in the 100-150 K range.
%The contours correspond to the integrated \threeohthree
%line intensity at levels [4,7,11,20,38] K \kms.
}
{fig:temmap}{0.6}{0}

\Figure{figures/big_lores_smoothdendro_tmap_withcontours}
{A temperature map created by averaging the dendrogram-derived clump temperature
along each line-of-sight.  See Figure \ref{fig:temmap} for details.}
{fig:dendrotemmap}{0.6}{0}



\section{Parameter Comparisons}
In the selected regions we averaged to acquire higher signal-to-noise for more
detailed modeling, we examined the correlation of various fitted parameters.

The 1D line width of the gas should be tightly correlated with the 3D Mach
number.  If turbulent heating is responsible for the observed gas temperatures,
there should also be a correlation between the Mach number of the turbulence
and the gas temperature.  However, Figure \ref{fig:temvsfwhm_regions} shows no
such correlation.

\Figure{figures/chi2_temperature_vs_linewidth_fieldsandsources}
{The fitted temperature as described in Section \ref{sec:linemodeling} plotted against
the fitted line width.  The blue symbols are compact `clump' sources and the red symbols
are large-area square regions.  No trend between the line width and temperature
is obvious.}
{fig:temvsfwhm_regions}
{0.5}{0}


\FigureTwo
{figures/simple/Brick_SW_fit_4_lines_simple_splinebaselined.pdf}
{figures/simple/G0.42+0.04box_fit_4_lines_simple_splinebaselined.pdf}
{Fits to the \para lines for (a) The Brick and (b) a box centered on G0.42+0.04.
These spectra have significantly different ratios and therefore derived temperatures:
The Brick has $T\gtrsim100$ K while G0.42+0.04 has $T\approx40-50$ K.
The bottom black spectrum is the residual, and the orange wiggly spectrum shows
the spline fit used to iteratively remove baseline ripples from the residual.}
{fig:twospectra_hotcold}{1}{3in}

\section{Discussion}
\subsection{Variation with Spatial Scale}
For any given cloud, the ratio \Rone varies with the spatial averaging scale.
For example, around `The Brick', the ratio decreases from $\sim0.55$ at the
position $\ell=0.237$, $b=0.008$ within a single spectrum to $\sim0.33$ when
averaged over a radius $\sim70\arcsec$ (Figure \ref{fig:brickradial}).
\todo{Does this indicate that temperatures are high in spatially compact regions,
and low elsewhere?  Or does this indicate a strong density selection bias?}

\Figure{figures/brick_examples/ratio_vs_scale.pdf}
{Radial plots centered on the southwest portion of `The Brick'.
The top panel shows the ratio \Rone as a function of aperture size, starting
from a single pixel.  A higher $R_1$ generally indicates a higher temperature.
The bottom panel shows the line width $\sigma = FWHM/2.35$
as a function of aperture size.}
{fig:brickradial}{0.6}{0}


\subsection{Disagreement between gas and dust temperatures}
The gas and dust temperatures throughout the CMZ continue to disagree far from
the central $\sim30$ pc previously shown in \citet{Ao2013a}.
Figure \ref{fig:temvstem_regions} shows the fitted \para temperature vs the fitted
dust temperature from HiGal.  The majority of the \formaldehyde data points are well
above the $T_{dust}=T_{gas}$ line.

\Figure{figures/chi2_temperature_vs_higaltemperature_fieldsandsources}
{The fitted temperature as described in Section \ref{sec:linemodeling} plotted
against the HiGal fitted dust temperature.  As in Figure \ref{fig:temvsfwhm},
the blue symbols are compact `clump' sources and the red symbols are large-area
square regions.  The black dashed line shows the $T_g = T_d$ relation.  Nearly
all of the data points fall above this relation, and very few are consistent
with it at the 3$\sigma$ level.}
{fig:temvstem_regions}
{0.5}{0}

\subsection{Discussion of Heating Mechanisms}
\citet{Ao2013a} examined four heating mechanisms in the inner $\sim 40$ pc,
concluding that the only viable heating mechanisms capable of explaining the
high observed temperatures in the molecular gas are cosmic ray and turbulent
heating.  

The cosmic ray heating rate required by the \citet{Ao2013a} analysis is high,
$\zeta\sim1-2\ee{-14}$ \pers, but plausible.  The turbulence in the CMZ is also
enough to explain the observed gas temperatures. 
The analysis performed in \citet{Ao2013a} determined the gas temperature
assuming that the observed temperature corresponds to an equilibrium between
heating by a single mechanism for all of the gas mass.  For cosmic ray-driven
heating, this is a reasonable approximation: the CR-derived gas temperature
has a weak density dependence ($T_{kin}\sim n^{1/6}$) and the energy deposited
by cosmic rays should be relatively uniform per molecule.  Turbulent heating,
on the other hand, is nonuniform \citep{Pan2009a}.

Non-uniform heating of the molecular gas could explain the temperature
structure observed in the CMZ molecular clouds.  If supersonic shocks within
the gas are responsible for heating the gas above a mean background temperature
$T_{gas} \sim T_{dust}$, only about 10\% of the gas needs to be heated to
$T\sim50-100$ K to generate the line brighntess observed \emph{if} the hottest
gas is also the densest.  However, hot and dense gas cools more efficiently, so
sustaining a population of shock-heated gas may be infeasible.  Additionally,
the log-poisson distribution followed by a multiplicative cascade has a substantial
low-temperature tail \citep{Pan2009a}, which implies that the mean temperature
must be near the observed temperature.

% However, it is still possible
% to distinguish cosmic ray and turbulent heating if one dominates over the
% other: turbulent heating depends strongly on the velocity dispersion ($\Gamma
% \propto \sigma_V^3$), while CR heating should have no dependence on the
% velocity dispersion.  Additionally, turbulence has a scale dependence, since
% the global dynamics (rotation) dominates over turbulence on scales larger than
% the scale height in the CMZ: $h\sim10$ pc (4\arcmin) in the 100-pc `ring',
% $h\sim50$ pc (20\arcmin) in the $\ell=1.3\arcdeg$ cloud \citep{Kruijssen2013a}.
% I'm not sure this really makes sense.  Is it even possible to measure the gas
% temperature on larger physical/angular scales?

% These differences result in testable predictions.  If turbulence is the
% dominant heating mechanism in the CMZ clouds, there should be a correlation
% between the observed line width and the gas kinetic temperature.

\todo{This is speculative.}  In a highly turbulent medium, most of the power
dissipation happens on the smallest size scale (if $L > \sigma_v^3$).  If the
cooling rate is enhanced in the postshock medium, the high gas temperature will
be isolated to high-density, compact clumps, but these clumps will also
dominate the line emission.  \todo{This should result in a density dependence
of the temperature or observed temperature $T(n) > n^1$, maybe $T(n) \sim n^2$ ?}
Cosmic rays should not exhibit this behavior; they should heat all scales equally
and care only about the \emph{mean} density.

\subsection{Shocks}
Assuming the dust temperature reflects the mean gas temperature and only a subset
of the gas has been heated to the high observed temperatures, we can determine the
shock velocity required to supply the heating.  Assuming an adiabatic strong shock,
$T_{ps} = \frac{3}{16}\frac{\mu v_s^2}{k}$, where $\mu = \mu_{\hh} = 2.8$.  A 1 \kms
shock can provide a temperature of 65 K

% NOTE: the baselines in The Brick are particularly nasty.  The most advanced
% fitting techniques will be required to extract anything useful.
\todo{Move to discussion:}
\citet{Johnston2014a} noted a very high temperature peak, $T>320$K, using these
lines of \formaldehyde toward The Brick, G0.253+0.015.  We measure similar line
ratios toward The Brick, but on larger scales.  The interpretation of these
line ratios is subject to some question, though.  Above 300K, the collision
rates provided by LAMDA \citep{Schoier2005a,Wiesenfeld2013a} must be
extrapolated.  Additionally, some of the ratios cannot be reproduced by high
temperatures, but instead require that the lines be thermalized and therefore
at very high densities.   Given the low brightness temperatures observed,
the emission must be coming from an extremely small area, which is inconsistent
with the observed large extent of the emission unless it is also extremely
patchy.  That remains a possibility if we are observing massive, dense clouds
with relatively uniformly turbulent properties.

%\section{Examination of Individual Spectra}
%%\todo{The modeling is ambiguous still: it is necessary to impose some sort of
%%restriction on the abundance or column density in order to acquire reasonable
%%constraints on the temperature.}
%In order to maximize the signal-to-noise and acquire the strongest constraint
%on the temperature, we extract spectra for regions with bright lines.  This
%approach is similar to the approach adopted in \citet{Ao2013a}, but for a much
%larger area.
%
%\subsection{Sources in the Ring}
%The \citet{Molinari2011a} ring has attracted a great deal of attention recently,
%as it has the most gas in the CMZ and a large, but unrealized, star-forming potential
%\citep{Longmore2013a,Longmore2012b,Longmore2012a,Kruijssen2013a,Yusef-Zadeh2009a,Immer2012a}.
%The star formation in this region is suppressed relative to expectations based
%on nearby star-forming regions and nearby (on a different scale) Galactic
%disks.  Hypotheses for this depressed star formation rate include suppressed
%fragmentation by cosmic ray and X-ray heating and a raised star formation
%threshold from enhanced turbulence.
%
%\subsubsection{The Brick: G0.253+0.016}
%We examine two lines of sight through The Brick, in the northeast at
%G0.241+0.006 and in the southwest at G0.261+0.028.  The northeast line of sight
%has two independent velocity components with dramatically different
%temperatures: a 30 \kms component with only a lower limit on temperature,
%$T>300$ K, and a 0 \kms component with $T=73\pm3$ K.  The southwest line of
%sight has only one component with $T=157\pm10$ K.
%
%\subsubsection{Cloud d: G0.38+0.04}
%Cloud ``d'' is one of the possible protoclusters identified in
%\citet{Longmore2013a}.  It exhibits bright and relatively narrow (FWHM$\sim7.3$
%\kms) \para emission and has a tightly constrained temperature $T=84\pm5$ K.
%
%\Figure{figures/G0.38+0.04_fit_h2co_mm_radex.pdf}
%{The 218-219 GHz spectrum of Cloud d}
%{fig:cloudDspec}{0.5}{0.0}
%
%\subsection{Clouds e/f: G0.47+0.01}
%Clouds ``e'' and ``f'' together make another proto-cluster in
%\citet{Longmore2013a}.  These clouds are marginally warmer than cloud d, with
%$T=110\pm20$ K.  However, because of the region of parameter space the lines
%allow, the density is reasonably constrained to be $n(\hh)\sim10^{4\pm0.2}$ \percc,
%much lower than in cloud d and more comparable to The Brick.

\section{Conclusion}


\textbf{Acknowledgements}:

\textbf{Code Packages Used}:

\begin{itemize}
    \item sdpy \url{https://github.com/keflavich/sdpy}
    \item aplpy \url{http://aplpy.github.io}
    \item pyradex \url{https://github.com/keflavich/pyradex}
\end{itemize}

\appendix{Baseline Removal}
The baselines in our SHFI-1 data were particularly problematic, more than is
usual in modern heterodyne obervations.  The 218 GHz window we have observed is
particularly sensitive to resonances within the SHFI-1 receiver that vary on
$<1$ minute timescales; our off-position calibrations were performed about once
per minute and therefore were not rapid enough to mitigate this problem
completely.  The baselines can broadly be described as smoothly varying ripples
on the scale of $\sim1/10$ the spectral window, plus more rapidly varying
ripples on 20-40 \kms scales.  In principle, this is a straightforward problem
of identifying the fourier components associated with each of these scales and
subtracting them.

In practice, we discovered that it was not possible to remove the dominant
baseline structure on either scale without significantly affecting the
underlying spectral data.  We simulated a variety of fourier-space suppression
approaches by adding synthetic signal to baseline spectra extracted from the
first few PCA components of the real spectra.

The PCA extraction approach is able to pull out the dominant baseline
components very effectively, but it inevitably includes significant signal in
the top few most correlated components, especially for the strong \formaldehyde
and \thirteenco lines.  We therefore abandoned it for the final data reduction.

For reference, we show an example spectrum that we believe to consist entirely
of baseline ripples in Figure \ref{fig:badbaselines}.

\Figure{figures/worst_baselines_map001_withsynth}
{An example showing some of the worst baselines observed.  The plotted spectrum
is from an observation on April 2, 2014, showing the average of the 5\% worst
spectra.  The blue curve shows a $\sigma=5$ \kms (FWHM$=11.75$ \kms) line centered
at the 0 \kms position of \para \threeohthree,
illustrating that the baseline `ripples' have widths comparable to the observed
lines.  While we selected the worst 5\% in this case, nearly all spectra are
affected by these sorts of baselines, and the shape and amplitude varies
dramatically and unpredictably.  The variation, unpredictable though it is,
works in our favor as it averages out over multiple independent observations.
The 218 GHz region shown here is also the worst-affected; the 220 GHz range
that includes the \thirteenco lines generally exhibits smoother and
lower-amplitude baseline spectra.}
{fig:badbaselines}{0.6}{0}

%\Figure{figures/M-091.F-0019-2013-2013-06-08_PCA_high_diagnostic.png}
%{The first 3 eigenspectra from an observation taken on June 8, 2013.  These
%show the baselines that are removed by PCA cleaning.  They do not include any
%signal from the spectral lines.  \todo{This will be
%replaced with a higher-quality figure}}
%{fig:eigenspectra}{0.5}{0}


\ifstandalone
\bibliographystyle{apj_w_etal}  % or "siam", or "alpha", or "abbrv"
\bibliography{bibdesk}      % bib database file refs.bib
\fi

\end{document}

