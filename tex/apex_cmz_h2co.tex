\input{preface}

\title{APEX CMZ Survey}
\newcommand{\eso}     {$^{1 }$}
\newcommand{\mpifr}   {$^{2 }$}
\newcommand{\saudi}   {$^{3 }$}
\newcommand{\naoj}    {$^{4 }$}
\newcommand{\pmo}     {$^{5 }$}
\newcommand{\nrao}    {$^{6 }$}
\newcommand{\casa}    {$^{7 }$}
\newcommand{\cfa}     {$^{8 }$}
\newcommand{\chalmers}{$^{9 }$}
\newcommand{\oxford}  {$^{10}$}
\newcommand{\mpa}     {$^{11}$}
\newcommand{\ljmu}    {$^{12}$}
\newcommand{\lmu}     {$^{13}$}

\author{Adam Ginsburg{\eso},
        Christian Henkel{\mpifr,\saudi},
        Yiping Ao{\naoj,\pmo},
        Denise Riquelme{\mpifr},
        Jens Kauffmann{\mpifr},
        Thushara Pillai{\mpifr},
        Elizabeth A.C. Mills{\nrao},
        M. A. Requena-Torres{\mpifr},
        Katharina Immer{\eso},
        Leonardo Testi{\eso},
        Juergen Ott{\nrao},
        John Bally{\casa},
        Cara Battersby{\cfa},
        Jeremy Darling{\casa},
        Susanne Aalto{\chalmers},
        Thomas Stanke{\eso},
        Sarah Kendrew{\oxford},
        J.M. Diederik Kruijssen{\mpa},
        Steven Longmore{\ljmu},
        James Dale{\lmu},
        %Alexandre Faure{}, % invite him first
        Rolf Guesten{\mpifr},
        K.M. Menten{\mpifr}
        }

%\institute{
%      {$^\casa$}{\it{CASA, University of Colorado, 389-UCB, Boulder, CO 80309}}}
%      {$^\eso$}{\it{European Southern Observatory, Karl-Schwarzschild-Strasse 2, D-85748 Garching bei München, Germany}}}
%      {$^\cfa$}{\it{CfA}}}
%      {$^\mpifr$}{\it{Max Planck Institute for Radio Astronomy, auf dem Hugel, Bonn}}}
%      {$^\nrao$}{\it{National Radio Astronomy Observatory, Socorro}}}
%      {$^\oxford$}{\it{Oxford}}}
%      {$^\chalmers$}{\it{Chalmers}}}
%}
\institute{
    {\eso}{\it{European Southern Observatory, Karl-Schwarzschild-Strasse 2, D-85748 Garching bei München, Germany\\
                      \email{Adam.Ginsburg@eso.org}}} \\ 
    {\mpifr}{\it{Max Planck Institute for Radio Astronomy, auf dem Hugel, Bonn}}
    {\saudi}{\it{Astron. Dept., King Abdulaziz University, P.O. Box 80203,
    Jeddah 21589, Saudi Arabia}}\\
    {\naoj}{National Astronomical Observatory of Japan, 2-21-1 Osawa, Mitaka, Tokyo 181-8588, Japan}
    {\pmo}{Purple Mountain Observatory, Chinese Academy of Sciences, Nanjing 210008, China}
    {\nrao}{\it{National Radio Astronomy Observatory, Socorro}}
    {\casa}{\it{CASA, University of Colorado, 389-UCB, Boulder, CO 80309}} \\ 
    {\cfa}{\it{Harvard-Smithsonian Center for Astrophysics, 60 Garden
    Street, Cambridge, MA 02138, USA}} \\ 
    {\chalmers}{\it{Department of Earth and Space Sciences,
                    Chalmers University of Technology}}
    {\oxford}{\it{Department of Astrophysics, The Denys Wilkinson Building, Keble Road, Oxford OX1 3RH}}
    %{\edmonton}{\it{University of Alberta, Department of Physics, 4-181 CCIS, Edmonton AB T6G 2E1 Canada}} \\ 
    %{\yale}{\it{Department of Astronomy, Yale University, P.O. Box 208101, New Haven, CT 06520-8101 USA}} \\ 
    %{\puertorico}{\it{Department of Physical Sciences, University of Puerto Rico, P.O. Box 23323, San Juan, PR 00931}}
    {\mpa}{\it{Max-Planck Institut f\"{u}r Astrophysik, Karl-Schwarzschild-Stra\ss e 1, 85748 Garching, Germany}}
    {\ljmu}{\it{Astrophysics Research Institute, Liverpool John Moores
    University, IC2, Liverpool Science Park, 146 Brownlow Hill, Liverpool L3
    5RF, United Kingdom}}
    {\lmu}{\it{University Observatory Munich, Scheinerstr. 1, D-81679 München, Germany}}
    }


\section{Abstract}

\footnote{Compiled on \today\ at \currenttime}


\section{Introduction}
\todo{To do.  To-do items are coded in red.}




\section{Observations \& Data Reduction}

\subsection{Observations}
We observed the Central Molecular Zone from $-0.4 < \ell < 1.6$ with the SHFI-1
instrument \citep{Vassilev2008a} on the APEX telescope using the XFFTS backend.
The observations were divided into 25 hours in June 2013 and 75 hours in
April-July 2014.  

The 2013 observations were taken in $4\arcmin \times 4\arcmin$ patches, and the
frequency range covered was 217.5-220 GHz and 216-218.5 GHz in the two spectral
windows.  Scans were performed along lines of constant RA and Dec.

The 2014 observing strategy was modified to use larger $8\arcmin \times
8\arcmin$ scans along lines of Galactic latitude and longitude.  The frequency
range was also shifted to cover windows over 217-219.5 and 218.4-220.9 GHz, thus
including the bright \thirteenco and \ceighteeno 2-1 lines.

The raw data were acquired with 32768 spectral channels in each window, yielding
0.1 \kms resolution.  We downsampled the data by a factor of 8 to 0.8 \kms
resolution to make the data more manageable.  We also expect to see no lines
narrower than a few \kms in the CMZ, particularly not with the relatively
shallow observations we have acquired.

\subsection{Reduction}
\subsubsection{Calibration}
Calibration was performed at the telescope using the standard APEX calibration
tools
\footnote{See the observing manuals:
\url{www.apex-telescope.org/documents/public/APEX-MPI-MAN-0012.pdf}
\url{www.apex-telescope.org/documents/public/APEX-MPI-MAN-0013.pdf} }.
These yield flux-calibrated spectra at each position with appropriate pointing
information.  Typical flux calibration uncertainties are $\sim$ \todo{???} and
pointing errors \todo{???} ($\sigma<2$\arcsec ?).

\subsubsection{Flagging bad spectra}
Spectra were removed if they showed excessive noise compared to the
theoretically expectation given the measured system temperature.  As in
\citet{Ao2013a}, the threshold was set to $1.5\times$ the theoretical noise,
i.e. $1.5 \sqrt{2} T_{sys} (\Delta\nu t_{exp})^0.5$, where $t_{exp}$ is the
exposure time (integration time) per spectrum in seconds and $\Delta\nu$ is the
channel width in Hz.  This approach resulted in $\sim0.2\%$ of the data being
removed.  

At the position of Sgr B2, the noise was significantly higher due to signal
from the continuum source.  We therefore disabled flagging in a
$\sim2.5\arcmin$ box around Sgr B2.


% Removing these baselines by
% the usual technique of masking out the emitting spectral region and fitting a
% low-order polynomial to the surroundings is somewhat impractical in the
% Galactic Center, where baseline ripples are on comparable scales to the
% extremely wide spectral lines.

%This method was abandoned: it resulted in negative bowls
% We perform a scan-by-scan baseline subtraction as follows: 
% \begin{enumerate}
%     \item Compute the mean spectrum across the scan
%     \item Identify pixels $>1-\sigma$ above the mean spectrum to be ignored
%         when baseline fitting (this is a simple `line masking' process)
%     \item Fit a line to each spectral channel across the scan
%     \item Smooth the fitted line parameters with a gaussian with with
%         $\sigma=10$ channels, interpolating into the line-masked regions
%     \item Subtract the fitted baselines
% \end{enumerate}
% 
% We therefore adopted a principle component analysis (PCA) baseline removal
% approach in which the data were spectrally smoothed and downsampled by an
% additional factor of 25, for a characteristic frequency scale of 15 MHz.  The
% data had their means subtracted before PCA cleaning so that only time-variable
% baseline variation was removed.  The resulting data were decomposed into
% eigenspectra, of which the 3 most correlated components were then set to zero.
% These components visually match the baseline ripples, and due to the wide
% smoothing (20 \kms resolution), no line information is lost.  The selection of
% 3 components was determined empirically by examining the eigenspectra; a test
% removing the first 10 components showed no significant difference in the
% cleaning but did show signs of extracting large-scale correlated signal.
% 
% \Figure{figures/M-091.F-0019-2013-2013-06-08_PCA_high_diagnostic.png}
% {The first 3 eigenspectra from an observation taken on June 8, 2013.  These
% show the baselines that are removed by PCA cleaning.  They do not include any
% signal from the spectral lines.  \todo{This will be
% replaced with a higher-quality figure}}
% {fig:eigenspectra}{0.5}{0}


\subsubsection{Mapmaking}
The maps were made by computing an output grid in Galactic coordinates with
10\arcsec pixels and adding each spectrum to the appropriate pixel\footnote{We
use the term `pixel' to refer to a square data element projected on the sky
with axes in Galactic coordinates.  The term `voxel' is used to indicate a cubic data
element, with two axes in galactic coordinates and a third in frequency or
velocity}.  In order
to avoid empty pixels and maximize the signal-to-noise, the spectra were added
to the grid with a weight set from a gaussian with $FWHM=10\arcsec$, which
effectively smooths the output images from $FWHM\approx28\arcsec$ to
$\approx30\arcsec$.  See \citet{Mangum2007a} for more detail on the on-the-fly
mapping technique used here.

%\todo{To demonstrate the utility of the PCA subtraction approach described in Section
%\ref{sec:baseline}, we show maps of the \formaldehyde \threeohthree line before
%and after PCA subtraction.}

The PPV cubes were created with units of brightness temperature ($T_A^*$).  The
main beam efficiency is $\eta_{mb} = 0.75$ (gain $\sim39$
Jy/K)\footnote{\url{http://www.apex-telescope.org/telescope/efficiency/}}.


\section{Signal Extraction}
\label{sec:signal}
We use the method described partially in \citet{Ao2013a} and more thoroughly in
\citet{Dame2011b} to mask the data cubes at locations of significant signal in 
the brightest line. 
A noise map was created by computing the sample standard deviation over a
200-pixel range in which no signal was identified.
We use the \formaldehyde \threeohthree line to create the mask by
the following procedure:

\begin{enumerate}
    \item Smooth the data with a gaussian of width 2 pixels in each direction
        (spatial and spectral)
    \item Identify all pixels with brightness $T_A > 3\sigma$, where the noise
        map was \emph{not} smoothed
    \item Grow the mask from the previous step by 1 pixel in each direction
        (this is known in image processing as binary dilation)
\end{enumerate}

The \formaldehyde \threeohthree mask was then applied to the \threetwoone and
\threetwotwo cubes, and all three cubes were integrated.  There is some overlap
between the \methanol \fourtwotwo line and the \formaldehyde \threetwotwo
line in PPV space, so we shifted the \formaldehyde mask to the velocity of the
\methanol line in the \formaldehyde \threetwotwo cube and excluded any pixels
expected to have signal.  There were XXX voxels excluded in this fashion.


\subsection{Baselining}
\label{sec:baseline}
The data showed significant baseline structure, leading to large-scale
correlated components in the resulting spectra.  The baselines were removed
after cube generation by first identifying bright regions in the \para
\threeohthree line using the technique described in Section \ref{sec:signal},
masking those out for each other line, then fitting a 5th-order polynomial to
the remaining spectra over the velocity range -150 to 250 \kms and subtracting
it.

After performing this baselining, the signal was re-extracted

\section{Ratio and Temperature Maps}
\label{sec:h2co}
We created two independent ratio maps, \threeohthree/\threetwotwo and
\threeohthree/\threetwoone, so that the two ratios (which are expected to be
the same) can be used as a sanity check.  However, in many regions, especially
those with the greatest signal, the \threetwotwo line is strongly affected by
contamination from a \methanol line, so the ratio is not particularly reliable.

\subsection{Computing Temperature}
\todo{Unless there's a strong reason not to, I will use the same grid that
Yiping used.}
We use RADEX \citet{van-Der-Tak2007a} to create a model grid for the
p-\formaldehyde lines over a sparse grid of density ($n=10^{4,5,6,7}$ \percc),
a fixed assumed line gradient of 5 \kms / pc, a fixed column of \formaldehyde
$N(\formaldehyde) = 9.5\ee{13}$ \persc (which corresponds to an abundance
$X=10^{-8.5}$ at $N(\hh) = 3\ee{22}$ \persc, and a range of temperatures from
10 to 300 K with 100 evenly spaced grid points.  The grid was created using
the \texttt{pyradex} wrapper of
RADEX\footnote{\url{https://github.com/keflavich/pyradex}}.

The line brightnesses returned from RADEX were then divided to create ratios,
and the grid inverted to allow interpolation of a measured ratio onto the input
temperature grid.   This approach requires that the ratios be monotonic with
temperature, which generally holds.  It also required us to \emph{a priori} select
a density from the grid; we selected $n=10^4$ \percc.  \todo{It is important to
assess what systematic effect density may have: to this end, I propose to show
temperature maps computed with (1) a $10\times$ higher density and (2) a
$10\times$ lower abundance.}.

\citet{Johnston2014a} noted a very high temperature peak, $T>320$K, using these
lines of \formaldehyde toward The Brick, G0.253+0.015.  We measure similar line
ratios toward The Brick, but on larger scales.  The interpretation of these
line ratios is subject to some question, though.  Above 300K, the collision
rates provided by LAMDA \citep{Schoier2005a,Wiesenfeld2013a} must be
extrapolated.  Additionally, some of the ratios cannot be reproduced by high
temperatures, but instead require that the lines be thermalized and therefore
at very high densities.   Given the low brightness temperatures observed,
the emission must be coming from an extremely small area, which is inconsistent
with the observed large extent of the emission unless it is also extremely
patchy.  That remains a possibility if we are observing massive, dense clouds
with relatively uniformly turbulent properties.

\FigureTwoAA
{figures/big_H2CO_322221_to_303202_smooth_bl_integ.pdf}
{figures/big_H2CO_321220_to_303202_smooth_bl_integ.pdf}
{Maps of the ratio of the integrated \para \threetwotwo / \threeohthree (top)
and \threetwoone/\threeohthree (bottom).  The data cubes were masked by
signal-to-noise ratio in the \threeohthree line, with blank (gray) regions
indicating nondetections.  The \threetwotwo line is also masked where it
overlaps with the \methanol \fourtwotwo line, resulting in a large masked area
around Sgr B2 in the top panel.  The Sgr B2 peaks also exhibit \formaldehyde absorption and therefore
bias the line ratios.  Higher ratios (red) correspond to higher
temperatures and/or densities.  \todo{Further modeling is required to determine
which.}
}
{fig:ratiomaps}{1}{7in}

\section{Line Modeling}
In order to extract more detailed information about the gas properties and
acquire a more detailed understanding of the uncertainties in the ratio-derived
temperature map described above, we extracted spectra from individual regions
with relatively high signal-to-noise and modeled them in further detail.

The RADEX grids described above were used, but with finer spacing in density,
column density, and temperature.  We then upsampled the grids by spline
interpolating between grid points to acquire a high-resolution ($250^3$) grid
covering $10<T<350$ K, $10^{2.5} < n < 10^7$ \percc, and $10^{11} < N <
10^{15.1}$ \persc\perkmspc.

We created $\chi^2$ grids using independent constraints from the line ratio,
the \formaldehyde abundance, and the total column density of \hh.  We use the
line \emph{ratio} rather than line brightnesses to avoid uncertainties due to
the ``filling factor'' of the emitting gas: even for diffuse clouds, the
filling factor of the emitting regions may be $ff<<1$ if the emission is
isolated to compact shocked regions, as expected if highly supersonic
$\mathcal{M}>10$ turbulence is energetically dominant in the clouds.  We use an
abundance $N(\para)/N(\hh) = X(\para) = 10^{9.08\pm1} = 1.2\ee{9} \times/\div
10$, allowing for dramatic uncertainty in the \formaldehyde abundance
\citep{Ginsburg2013a,Carey1998a,Wootten1978a,Mundy1987a}.  To constrain the
total column density, we use the Herschel dust maps to derive an \hh column
density, which has a nominal $\sim2-3\times$ uncertainty.  We treat the erro as
$10\times$ to account for the abundance uncertainty again.

These constraints are shown projected onto the three planes of our fitted
parameter space in the multi-paneled Figures
\ref{fig:coltemconstraints}-\ref{fig:parsonbrightness}.

\Figure{figures/d:G0.38+0.04_fit_4_lines_simple.pdf}
{Spectrum of ``cloud d"}
{fig:clouddspec}{0.5}{0}
\Figure{figures/param_fits/d:G0.38+0.04_col_tem_parameter_constraints.pdf}
{caption}
{fig:coltemconstraints}{0.5}{0}
\Figure{figures/param_fits/d:G0.38+0.04_dens_col_parameter_constraints.pdf}
{caption}
{fig:denscolconstraints}{0.5}{0}
\Figure{figures/param_fits/d:G0.38+0.04_dens_tem_parameter_constraints.pdf}
{caption}
{fig:denstemconstraints}{0.5}{0}
\Figure{figures/param_fits/d:G0.38+0.04_h2coratio_minaxis.pdf}
{caption}
{fig:parsonbrightness}{0.5}{0}

\section{Examination of Individual Spectra}
\todo{The modeling is ambiguous still: it is necessary to impose some sort of
restriction on the abundance or column density in order to acquire reasonable
constraints on the temperature.}
In order to maximize the signal-to-noise and acquire the strongest constraint
on the temperature, we extract spectra for regions with bright lines.  This
approach is similar to the approach adopted in \citet{Ao2013a}, but for a much
larger area.

\subsection{Sources in the Ring}
The \citet{Molinari2011a} ring has attracted a great deal of attention recently,
as it has the most gas in the CMZ and a large, but unrealized, star-forming potential
\citep{Longmore2013a,Longmore2012b,Longmore2012a,Kruijssen2013a,Yusef-Zadeh2009a,Immer2012a}.
The star formation in this region is suppressed relative to expectations based
on nearby star-forming regions and nearby (on a different scale) Galactic
disks.  Hypotheses for this depressed star formation rate include suppressed
fragmentation by cosmic ray and X-ray heating and a raised star formation
threshold from enhanced turbulence.

\subsubsection{The Brick: G0.253+0.016}
We examine two lines of sight through The Brick, in the northeast at
G0.241+0.006 and in the southwest at G0.261+0.028.  The northeast line of sight
has two independent velocity components with dramatically different
temperatures: a 30 \kms component with only a lower limit on temperature,
$T>300$ K, and a 0 \kms component with $T=73\pm3$ K.  The southwest line of
sight has only one component with $T=157\pm10$ K.

\subsubsection{Cloud d: G0.38+0.04}
Cloud ``d'' is one of the possible protoclusters identified in
\citet{Longmore2013a}.  It exhibits bright and relatively narrow (FWHM$\sim7.3$
\kms) \para emission and has a tightly constrained temperature $T=84\pm5$ K.

\Figure{figures/G0.38+0.04_fit_h2co_mm_radex.pdf}
{The 218-219 GHz spectrum of Cloud d}
{fig:cloudDspec}{0.5}{0.0}

\subsection{Clouds e/f: G0.47+0.01}
Clouds ``e'' and ``f'' together make another proto-cluster in
\citet{Longmore2013a}.  These clouds are marginally warmer than cloud d, with
$T=110\pm20$ K.  However, because of the region of parameter space the lines
allow, the density is reasonably constrained to be $n(\hh)\sim10^{4\pm0.2}$ \percc,
much lower than in cloud d and more comparable to The Brick.

\section{Conclusion}


\textbf{Acknowledgements}:

\textbf{Code Packages Used}:

\begin{itemize}
    \item sdpy \url{https://github.com/keflavich/sdpy}
    \item aplpy \url{http://aplpy.github.io}
    \item pyradex \url{https://github.com/keflavich/pyradex}
\end{itemize}


\ifstandalone
\bibliographystyle{apj_w_etal}  % or "siam", or "alpha", or "abbrv"
\bibliography{bibdesk}      % bib database file refs.bib
\fi

\end{document}

